% !TeX program = pdfLaTeX
\documentclass[smallextended]{svjour3}       % onecolumn (second format)
%\documentclass[twocolumn]{svjour3}          % twocolumn
%
\smartqed  % flush right qed marks, e.g. at end of proof
%
\usepackage{amsmath}
\usepackage{graphicx}
\usepackage[utf8]{inputenc}

\usepackage[hyphens]{url} % not crucial - just used below for the URL
\usepackage{hyperref}
\providecommand{\tightlist}{%
  \setlength{\itemsep}{0pt}\setlength{\parskip}{0pt}}

%
% \usepackage{mathptmx}      % use Times fonts if available on your TeX system
%
% insert here the call for the packages your document requires
%\usepackage{latexsym}
% etc.
%
% please place your own definitions here and don't use \def but
% \newcommand{}{}
%
% Insert the name of "your journal" with
% \journalname{myjournal}
%

%% load any required packages here



% Pandoc citation processing

\usepackage{gensymb}
\usepackage{subfig}
\usepackage{chngcntr}
\usepackage{natbib}

\begin{document}

\title{Assessment of zonal symmetric and asymmetric components of the Southern Annular Mode using a novel approach \thanks{The research was supported by UBACyT20020170100428BA and the CLIMAX Project funded by Belmont Forum/ANR-15-JCL/-0002-01. Elio Campitelli was supported by a PhD grant from CONICET, Argentina.} }


    \titlerunning{Assessment of zonal symmetric and asymmetric components of the SAM}

\author{  Elio Campitelli \and  Leandro B. Díaz \and  Carolina Vera \and  }


\institute{
        Elio Campitelli \at
     Universidad de Buenos Aires, Facultad de Ciencias Exactas y Naturales, Departamento de Ciencias de la Atmósfera y los Océanos, Buenos Aires, Argentina
 CONICET -- Universidad de Buenos Aires, Centro de Investigaciones del Mar y la Atmósfera (CIMA), Buenos Aires, Argentina
 CNRS -- IRD -- CONICET -- UBA, Instituto Franco‐Argentino para el Estudio del Clima y sus Impactos (UMI 3351 IFAECI), Buenos Aires, Argentina \\
     \email{\href{mailto:elio.campitelli@cima.fcen.uba.ar}{\nolinkurl{elio.campitelli@cima.fcen.uba.ar}}}  %  \\
%             \emph{Present address:} of F. Author  %  if needed
    \and
        Leandro B. Díaz \at
     Universidad de Buenos Aires, Facultad de Ciencias Exactas y Naturales, Departamento de Ciencias de la Atmósfera y los Océanos, Buenos Aires, Argentina
 CONICET -- Universidad de Buenos Aires, Centro de Investigaciones del Mar y la Atmósfera (CIMA), Buenos Aires, Argentina
 CNRS -- IRD -- CONICET -- UBA, Instituto Franco‐Argentino para el Estudio del Clima y sus Impactos (UMI 3351 IFAECI), Buenos Aires, Argentina \\
     %  \\
%             \emph{Present address:} of F. Author  %  if needed
    \and
        Carolina Vera \at
     Universidad de Buenos Aires, Facultad de Ciencias Exactas y Naturales, Departamento de Ciencias de la Atmósfera y los Océanos, Buenos Aires, Argentina
 CONICET -- Universidad de Buenos Aires, Centro de Investigaciones del Mar y la Atmósfera (CIMA), Buenos Aires, Argentina
 CNRS -- IRD -- CONICET -- UBA, Instituto Franco‐Argentino para el Estudio del Clima y sus Impactos (UMI 3351 IFAECI), Buenos Aires, Argentina \\
     %  \\
%             \emph{Present address:} of F. Author  %  if needed
    \and
    }

\date{Received: date / Accepted: date}
% The correct dates will be entered by the editor


\maketitle

\begin{abstract}
The text of your abstract. 150 -- 250 words.
\\
\keywords{
        Southern Annular Mode \and
        general circulation \and
        zonally asymmetric circulation \and
        El Niño Southern Oscillation \and
    }


\end{abstract}


\def\spacingset#1{\renewcommand{\baselinestretch}%
{#1}\small\normalsize} \spacingset{1}


\hypertarget{introduction}{%
\section{Introduction}\label{introduction}}

The Southern Annular Mode (SAM) is the main mode of variability in the Southern Hemisphere extratropical circulation \citep{rogers1982} on daily, monthly, and decadal timescales \citep{baldwin2001a, fogt2006} and exerts an important influence on temperature and precipitation anomalies, and sea ice concentration \citep[e.g.][]{fogt2020}. Its positive phase is usually described as anomalously low pressures over Antarctica surrounded by a ring of anomalous high pressures in middle-to-high latitudes.

Most authors describe the SAM as a zonally symmetric pattern, a fact that is reflected not only in its name, but also in the various methods used to characterise it. Of the several different indices presented in the literature, many of them are based on zonal means of sea level pressure or geopotential height \citep{ho2012}. \citet{gong1999} defined the SAM index as the zonal mean sea level pressure difference between 40\degree S and 60\degree S, which is also the definition used by the station-based index in \citet{marshall2003}. \citet{baldwin2009} proposed defining the Northern and Annular modes as the leading EOF of the zonally averaged geopotential height at each level.

Even though these indices are based on zonal averages, their associated geopotential height spatial anomalies contain noticeable deviations from zonal symmetry, particularly in the Pacific Ocean region. The zonal asymmetries have not been widely studied, but previous work suggest that they strongly modulate the regional impacts of the SAM \citep{fan2007, silvestri2009, fogt2012, rosso2018}. The fact that the SAM is not entirely zonally symmetric hinders our ability to reconstruct its historical variability prior to the availability of dense observations in the Southern Hemisphere \citep{jones2009}.

Some of the variability associated with the zonal asymmetries of the SAM seems to be forced by the tropics. ENSO-like variability affects the Southern Hemisphere extratopics through the Rossby wave trains \citep{mo1987, kidson1988, karoly1989} which project strongly onto the zonal anomalies associated with the SAM in the Pacific sector. Moreover, tropical influences on the SAM have been observed \citep{fan2007, fogt2011, clem2013}. \citet{fan2007} computed SAM indices of the Western and the Eastern Hemisphere separately and found that they were much more correlated to each other if the (linear) signal of the ENSO was removed.

Positive trends in SAM index have been documented by various researchers using different indices mostly on austral Summer and Autumn \citep[e.g.][ and references therein]{fogt2020}. These trends are thought of driven primarily by stratospheric ozone depletion and the increase in greenhouse gases and understood in the context of zonal mean variables \citep{marshall2004, gillett2005, arblaster2006, gillett2013}. However, it's not clear yet how or if the asymmetric SAM component responds to this forcing, or how its variability alters the observed trends.

The impact of the zonally asymmetric component of the SAM at regional scales has not been studied in detail yet. Positive phase of the SAM is associated with colder-than-normal temperatures over Antarctica and warmer-than-normal temperatures at higher latitudes \citep{jones2019} (and vice versa for negative SAM). But there are significant deviations from this zonal mean response, notably in the Antarctic Peninsula and the South Atlantic \citep{fogt2012}. SAM signal on precipitation anomalies behaves similarly, although with even greater deviation from zonal symmetry \citep{lim2016}. The importance of zonal asymmetries of the SAM on these impacts have been studied in certain regions. For example, the SAM-precipitation relationshipin Southeastern South America can be explained by the Pacific-South American (PSA)-like zonally asymmetric circulation associated with the SAM \citep{silvestri2009, rosso2018}. \citet{fan2007} also found that precipitation in East Asia was impacted by the variability of only the Western Hemisphere part of the SAM.

The study of the temporal variability of the asymmetric component of the SAM has not received much attention except for \citet{fogt2012}. This study provides evidences of the relevance of SAM asymmetric component. However, their conclusions are based on composites of positive and negative SAM events including a small number of cases unevenly distributed among years with and without satellite information. The latter is particularly important due to the inhomogeneities in reanalysis products prior to the satellite era and the possible change in the asymmetric structure of the SAM \citep{silvestri2009}. Moreover, \citet{fogt2012} studied the zonal asymmetric component of the SAM only in sea level pressure. Zonal asymmetries in the SAM spatial pattern are fairly barotropic throughout the troposphere, but they change dramatically in the stratosphere \citep{baldwin2009}.

Our objective is, then, to describe the zonally asymmetric and symmetric components of the SAM variability. We first propose a methodology that provides for each level, two indices which aim to capture independently the variability of the symmetric and asymmetric SAM component respectively. Their vertical structure and coherence, temporal variability and trends are consequently assesed. We then study the spatial patterns described by the variability exclusive to each index focusing on 50 hPa as representing the stratosphere and 700 hPa as representing the troposphere. Finally, the relationships of the SAM at 700 hPa with temperature and precipitation anomalies are investigated.

In the Section \ref{methods} we describe the methods. In Section \ref{temporal} we describe the temporal variability and vertical coherence of the indices. In Section \ref{spatial}, we analyse the spatial patterns of geopotential height associated with them. In Section \ref{impacts}, we study their relationship with surface-level temperature and precipitation.

\hypertarget{methods}{%
\section{Methods}\label{methods}}

\hypertarget{data}{%
\subsection{Data}\label{data}}

We used monthly geopotential height at 2.5\degree~longitude by 2.5\degree~latitude of horizontal resolution and 37 vertical isobaric levels from ERA5 \citep{hersbach2020} for the period 1979 to 2018. We restrict our analysis to the post-satellite era to avoid any confounding factors arising from the incorporation of satellite observations.

Temperature data used are from NOAA's Merged Land Ocean Global Surface Temperature Analysis V4.0.1 \citep{smith2008, vose2012}, which blends land surface air temperature and sea surface (water) temperature analysis into a monthly global grid 5\degree~longitude by 5\degree~latitude. Monthly rainfall fields at 0.5\degree~longitude by 0.5\degree~latitude data from the Global Precipitation Climatology Centre \citep{schneider2015, schneider2017} are also considered. The rainfall dataset is based on station-based records, and thus it only has continental coverage.

\hypertarget{definition-of-indices}{%
\subsection{Definition of indices}\label{definition-of-indices}}

Traditionally the SAM is defined as the leading empirical orthogonal mode (EOF) of sea-level pressure or geopotential height anomalies at low levels \citep{ho2012}. Following \citet{baldwin2001}, we extend that definition vertically and use the term SAM to refer to the leading EOF of the monthly anomalies of geopotential height south of 20\degree S at each level. We performed EOFs by computing the Singular Value Decomposition of the data matrix consisting in 481 rows and 4176 columns (144 points of longitude and 29 points of latitude). We weighted the values by the square root of the cosine of latitude to account for the non-equal area of each gridpoint \citep{chung1999}. We consider in the EOF analysis all months together without separating in seasons.

To separate the zonally symmetric and asymmetric components of the SAM, we computed the zonal mean and anomalies of the full SAM spatial pattern, as shown in Figure \ref{fig:method} at 700 hPa. The full spatial signal (\(\mathrm{EOF_1}(\lambda, \phi)\)) is the sum of the zonally asymmetric (\(\mathrm{EOF_1^*}(\lambda, \phi)\)) and symmetric (\([\mathrm{EOF_1}](\lambda, \phi)\)) components. We then compute the ``Full SAM'', ``Asymmetric SAM'' and ``Symmetric SAM'' indices as the regression coefficients of the regression of each monthly geopotential height field on the respective patterns (weighting by the cosine of latitude). The three indices are then normalized by dividing them by the standard deviation of the Full SAM index at each level. As a result, the magnitudes between indices are comparable. However, only Full SAM index has unit standard deviation per definition. The explained variance of each pattern is used as an indicator of the degree of zonally symmetry or asymmetry of each monthly field. To quantify the coherence between temporal series corresponding to different indices or the same index at different levels, we computed the temporal correlation between them.

\begin{figure*}
\includegraphics{method-1} \caption{Spatial patterns of the first EOF of 700 hPa geopotential height for 1979 -– 2018 period. (a) Full field, (b) zonally asymmetric component and (c) zonally symmetric component. Arbitrary units.}\label{fig:method}
\end{figure*}

The method assumes linearity in the asymmetric component of the SAM. That means that zonal symmetries associated with positive SAM phase (SAM+) are almost opposite in sign and of the same magnitude to the ones associated with negative SAM phase (SAM-). \citet{fogt2012}'s composites (their Figure 4) suggest that this might not be entirely valid, although much of that apparent non-linearity could be due to the heterogenous nature of the selected years for constructing the composites. To test this assumption, we computed seasonal composites of zonal anomalies of geopotential height for SAM+ and SAM- (defined as months in which the Full SAM index is greater than 1 standard deviation and lower than minus 1 standard deviation, respectively) for the period from 1979 to 2018 at the 700 hPa and 50 HPa levels (Figures \ref{fig:A3} and \ref{fig:A4}). In all seasons and both levels, SAM+ composites are similar to SAM- in structure but with the opposite sign. Spatial correlations between composites for each season are high. The method considered in this study seems then a reasonable approximation of the phenomenon.

By performing the EOF analysis using data for all months we are assuming that the zonal asymmetric structure of the SAM is the same at all seasons. The latter was assessed by computing geopotential height zonal anomalies by projecting the first EOF of each season independently. The following seasons were considered -- December to February (DJF), March to May (MAM), June to August (JJA) and September to November (SON). Results are very similar to each other in the troposphere (Figure \ref{fig:A5}, row 1) and show spatial correlations between 0.65 (DJF with JJA) and 0.9 (MAM with SON). In the stratosphere (Figure \ref{fig:A5}, row 2), patterns are similar for all seasons except DJF, when the wave-1 zonal anomalies are rotated 90\degree~in comparison with the rest of the year. Spatial correlations in the stratosphere are between -0.24 (DJF with SON) and 0.95 (MAM with JJA). Therefore, the results confirm that the zonal asymmetric structure of the SAM is very similar throughout most of the year with the exception of DJF in the stratosphere.

The method also implies that the zonally asymmetric pattern of SAM remains stationary along the period considered. \citet{silvestri2009} suggest that this might not be the case between 1958 and 2004. Zonal asymmetric patterns of SAM were computed for the two halves of the period (1979 to 1998 and 1999 to 2018) respectively. They show no systematic change neither in the stratosphere nor in the troposphere (Figure \ref{fig:A6}).

\hypertarget{regressions}{%
\subsection{Regressions}\label{regressions}}

We perform linear regressions to quantify the association between the SAM indices and other variables. Moreover, we apply multiple linear regression analysis to describe the combined influence of both Asymmetric and Symmetric SAM indices. To obtain the linear coefficients of a variable \(X\) (geopotential, temperature, precipitation, etc\ldots{}) with the Asymmetric SAM (\(SAM_a\)) and Symmetric SAM (\(SAM_s\)) we fit the equation

\[
X(\lambda, \phi, t) = \alpha(\lambda, \phi) SAM_a + \beta(\lambda, \phi) SAM_s + X_0(\lambda, \phi) +  \epsilon(\lambda, \phi, t)
\]

where \(\lambda\) and \(\phi\) are the longitude and latitude, \(t\) is the time, \(\alpha\) and \(\beta\) are the linear regression coefficients, \(X_0\) and \(\epsilon\) are the constant and error terms. From this equation, \(\alpha\) represents the (linear) association of \(X\) with the variability of the Asymmetric SAM that is not explained by the variability of the Symmetric SAM; i.e.~it is proportional to the partial correlation of \(X\) and the Asymmetric SAM, controlling for the effect of the Symmetric SAM and vice versa for \(\beta\). When performing a separate regression for each trimester (DJF, MAM, JJA, SON), we average the relevant variables seasonally for each year and trimester before computing the regression.

Statistical significance for regression fields were evaluated adjusting p-values by controlling for the False Discovery Rate \citep{benjamini1995, wilks2016} to avoid misleading results from a high number of regressions \citep{walker1914, katz1991}.

Linear trends were computed by Ordinary Least Squares and the 95\% confidence interval assuming a t-distribution of the appropriate residual degrees of freedom. To the amplitude of the zonal waves is defined through computing the Fourier transform of the spatial field at each latitude circle.

We computed density probability estimates using a gaussian kernel of optimal bindwidth according to \citet{sheather1991}.

\hypertarget{computation-procedures}{%
\subsection{Computation procedures}\label{computation-procedures}}

We performed all analysis in this paper using the R programming language \citep{rcoreteam2020}, using the data.table package \citep{dowle2020} and the metR package \citep{campitelli2020}. All graphics are made using ggplot2 \citep{wickham2009}. We downloaded data from reanalysis using the ecmwfr package \citep{hufkens2020} and indices of the ENSO with the rsoi package \citep{albers2020}. The paper was rendered using knitr and rmarkdown \citep{xie2015, allaire2019}.

\hypertarget{results}{%
\section{Results}\label{results}}

\hypertarget{temporal}{%
\subsection{Temporal evolution}\label{temporal}}

\begin{figure*}
\includegraphics{asymsam-timeseries-1} \caption{Time series for the Asymmetric SAM and Symmetric SAM indices at (a) 50 hPa and (b) 700 hPa. To the right, probability density estimate of each index. Series are standarised by the standard deviation of the Full SAM at each level.}\label{fig:asymsam-timeseries}
\end{figure*}

We first asses the temporal evolution of the Asymmetric SAM and Symmetric SAM. Figure \ref{fig:asymsam-timeseries} shows the corresponding time series for 700 hPa and 50 hPa and their corresponding density estimates. We selected these two levels as representative of the tropospheric and stratospheric variability respectively. As it is shown below, the variabilities of both indices are highly coherent within each atmospheric layer, therefore is reasonable to take one level as representative of each layer.

Month-to-month variability is evident for both indices, with noisy variations in the low frequency. At first glance the series can be distinguished by their distributions. Compared to the tropospheric indices, the stratospheric indices are much more long-tailed; that is, extreme values (both negative and positive) abound. The Asymmetric SAM series have both more variability in the higher frequencies than the Symmetric SAM series.

The stratospheric Symmetric SAM varies strongly with a two-year period, which can be seen by spectral analysis (Figure \ref{fig:A2}). This might suggest a link between stratospheric SAM variability and the Quasi-Biennial Oscillation \citep{baldwin2001b}. A local peak at 2 years years is discernible in the periodogram of the tropospheric Symmetric SAM, although it's not statistically significant. In the troposphere the most significant peak of variability is found in the Asymmetric index at around 3.6 months.

From a visual inspection, the Asymmetric SAM and Symmetric SAM time series appear to be correlated. Moreover, looking at the extremes in the stratosphere, the Symmetric SAM series appears to lag the Asymmetric SAM series (see, for example, the positive events on late 1987). Figure \ref{fig:cor-lev} shows these correlations along all levels considered, for zero and -1 lags. Values of zero-lag correlations between the Asymmetric SAM and Symmetric SAM series are relatively constant throughout the troposphere, fluctuating between 0.39 and 0.45. One-month-lag correlations are similarly constant but significantly reduced to around 0.17. In the stratosphere, zero-lag correlations drop to a minimum of 0.21 at 20 hPa and then it increases again monotonically with height up to the uppermost level of the reanalysis (although results near the top of the models are to be interpreted with care). At the same time, one-month-lag correlations increase with height. Therefore, stratospheric Asymmetric SAM index tends to precede the Symmetric SAM index.

\begin{figure}
\includegraphics{cor-lev-1} \caption{Correlation between the Symmetric SAM and Asymmetric SAM index at each level for lag zero and lag -1 (Asymmetric SAM leads Symmetric SAM) for the 1979 -- 2018 period.}\label{fig:cor-lev}
\end{figure}

\begin{figure*}
\includegraphics{cross-correlation-1} \caption{Cross correlation between levels of the (a) Full SAM, (b) Asymmetric SAM, and (c) Symmetric SAM for the 1979 -- 2018 period.}\label{fig:cross-correlation}
\end{figure*}

Figure \ref{fig:cross-correlation} shows (zero-lag) cross-correlation across levels for the Full, Symmetric and Asymmetric SAM indices. For the Full SAM (Figure \ref{fig:cross-correlation}a), high values below 100 hPa reflect the vertical (zero-lag) coherency throughout the troposphere. Above 100 hPa, correlation between levels does falls off more rapidly, indicating less coherent (zero-lag) variability. But correlations between tropospheric and lower-to-middle stratospheric levels are still relatively high (e.g.~grater than 0.4 between tropospheric levels and levels below 30 hPa). Asymmetric and Symmetric SAM (Figure \ref{fig:cross-correlation}b and c, respectively) share similar high level of coherency in the troposphere but they differ in their stratospheric behaviour. Stratospheric coherency is stronger for the Asymmetric SAM than the Symmetric SAM. The stratospheric Symmetric SAM seems to connect more strongly to the troposphere than the Asymmetric SAM.

\begin{figure*}
\includegraphics{trends-1} \caption{Linear trends (in standard deviations per decade) at each level for annual (row 1) and seasonal values (rows 2 to 5) for the period 1979 -- 2018 and for the (column a) Full SAM index, (column b) Asymmetric SAM index, and (column c) Symmetric SAM index. Shading indicates the 95\% confidence interval from a t-distribution.}\label{fig:trends}
\end{figure*}

The linear trends for each of the indices (Full SAM, Symmetric SAM and Asymmetric SAM) were evaluated for the complete period 1979 -- 2018 at each level for the whole year and separated by trimesters (Figure \ref{fig:trends}). The Full SAM index presents a statistically positive significant trend (panel a.1) that extends throughout the troposphere up to about 50 hPa and reaches its maximum value at 100 hPa. The seasonal trends (rest of Figure \ref{fig:trends} column a) indicate that positive trends are present in Autumn and particularly in Summer, where the 100 hPa maximum is much more defined. In Winter and Spring, we detect no statistically significant trend. This is consistent with the results of previous studies, which find large positive trends in Summer, smaller in Autumn and no trends in the other seasons \citep[e.g.][ and references therein]{fogt2020} using indices of the SAM based on surface or near-surface circulation.

By separating the SAM signal in its asymmetric and symmetric parts, we can not only see that these trends are almost entirely due to the symmetric component (column b vs.~column c in Figure \ref{fig:trends}), but in some cases the trends become clearer. In Summer, the Asymmetric SAM has a statistically non significant negative trend in the middle troposphere that obscures the trend in the Full SAM index; as a result, trends computed using only the Symmetric component are stronger (compare the shading region in panel a.2 and c.2). In Autumn, the Symmetric SAM reveals a statistically significant positive trend in the stratosphere that is not significant using the Full SAM index.

Figure \ref{fig:r-squared-trend} shows trends for the explained variance of each index. There is no evidence of a significant trend in the stratosphere. In the troposphere, there is a positive trend for the Asymmetric SAM and not significant trend for the Symmetric SAM. This suggest that the SAM has become more asymmetric in the period from 1979 to 2018. However, the change is slight, around 1\% increased explained variance per decade.

\begin{figure}
\includegraphics{r-squared-trend-1} \caption{Decadal trends of the variance explained by the Asymmetric and Symmetric SAM at each level for the period 1979 -- 2018. Shading indicates the 95\% confidence interval.}\label{fig:r-squared-trend}
\end{figure}

\hypertarget{spatial}{%
\subsection{Spatial patterns}\label{spatial}}

\begin{figure*}
\includegraphics{2d-regr-1} \caption{Regression of geopotential height (meters) at (row 1) 50 hPa and (row 2) 700 hPa with the (column a) Full SAM, (column b) Asymmetric SAM, and (column c) Symmetric SAM for the 1979 -- 2018 period. The regression patterns for Asymmetric and Symmetric SAM are the result of one multiple regression using both indices. Points marked on panel b.2 are the location of the reference points used by \cite{raphael2004} for their Zonal Wave 3 index. }\label{fig:2d-regr}
\end{figure*}

We then computed the spatial regression of geopotential height anomalies on Full, Asymmetric and Symmetric indices at 700 hPa and 50 hPa levels (Figure \ref{fig:2d-regr}). While, regression coefficients in column a are computed using the Full SAM, the regression coefficients in columns b and c of Figure \ref{fig:2d-regr} are computed using multiple regression using the Asymmetric and Symmetric indices at the same time. Thus, they are to be interpreted as the patterns associated with each index, removing the variability (linearly) explained by the other index.

In the stratosphere, the spatial pattern associated with the Full SAM is clearly dominated by a zonally symmetric, monopolar structure (Figure \ref{fig:2d-regr}a.1) which is, not centred in the South Pole. On the other hand, the monopole associated with the Symmetric SAM (Figure \ref{fig:2d-regr}c.1) is more symmetric. Furthermore, the regression pattern of the Asymmetric SAM is characterized by a wave-1 structure with centers over the Bellinghausen Sea on the
Western Hemisphere and Davis Sea in the Eastern Hemisphere

In the troposphere, the regression pattern associated with Full SAM shows the well-known combination of zonally symmetrical annular mode with zonal asymmetries in the form of a wave-3 (Figure \ref{fig:2d-regr}a.2, \citep{fogt2012}). The regression patterns associated with the Asymmetric and Symmetric SAM indices successfully disentangle both structures. While, the Asymmetric SAM index gives rise to a cleaner zonal wave (Figure \ref{fig:2d-regr}b.2), the Symmetric SAM index is associated with an annular structure, almost devoid of zonal asymmetries (Figure \ref{fig:2d-regr}c.2). The wave-3 pattern observed in Figure \ref{fig:2d-regr}b.2 is rotated by half a wavelength from the average position of the mean wave-3 pattern described by Raphael (2004), whose reference locations are marked with points in the figure. Thus, the tropospheric Asymmetric SAM index represents a zonal displacement in the position of the climatological wave-3 pattern.

\begin{figure*}
\includegraphics{wave-amplitude-1} \caption{Amplitude (meters) of zonal waves of the geopotential height regression patterns in Figure \\ref{fig:2d-regr} for zonal waves with wavenumber 0, 1, 2, and 3, where wavenumber 0 represents the amplitude of the zonal mean.}\label{fig:wave-amplitude}
\end{figure*}

The amplitude of the first zonal wave numbers at each latitude at 50 hPa and 700 hPa is shown in Figure \ref{fig:wave-amplitude}, where wave number zero represents the amplitude of the zonal mean. Zonal wave amplitudes of the spatial pattern described by the Full SAM index (Figure \ref{fig:wave-amplitude} column a) are dominated by the zonal mean (wavenumber 0) at both levels. However, zonal waves are important, particularly North of 50\degree S, with wavenumber 1 clearly dominating at 50 hPa (Figure \ref{fig:wave-amplitude}a.1) and a mix of waves of similar amplitude at 700 hPa (Figure \ref{fig:wave-amplitude}a.2). Figure \ref{fig:wave-amplitude} column b shows that the Asymmetric SAM is overwhelmingly dominated by wave 1 in the stratosphere (Figure \ref{fig:wave-amplitude}b.1), while in the troposphere it is explained by a combination of zonal waves 3 to 1 in decreasing level of importance (Figure \ref{fig:wave-amplitude}b.2) with negligible amplitude of the zonal mean. On the other hand, the Symmetric SAM it is almost entirely explained by the zonal mean at both levels (Figure \ref{fig:wave-amplitude} column c), with little to no contribution from the zonal waves with wavenumbers 1 to 3.

Looking at Figure \ref{fig:2d-regr}b.2, it becomes apparent that zonal waves 1 and 2 modulate the amplitude of zonal wave 3, which -- as mentioned before -- is larger in the Western Hemisphere than in the Eastern Hemisphere.

\begin{figure}
\includegraphics{vertical-regression-1} \caption{Regression between monthly geopotential height anomalies (meters) averaged betweeen 65\degree and 40\degree S and the Asymmetric SAM index (extracted from multiple regression including the Symmetric SAM). (a) With the Asymmetric SAM in 50 hPa and (b) in 700 hPa for the 1979 -- 2018 period.}\label{fig:vertical-regression}
\end{figure}

To analyse the vertical structure of the geopotential height anomalies associated with the asymmetric SAM index, we show a vertical cross section of regressions of mean geopotential height between 65\degree S and 40\degree S for the 50 hPa Asymmetric SAM index (Figure \ref{fig:vertical-regression}a) and for the 700 hPa Asymmetric SAM index (Figure \ref{fig:vertical-regression}b). The geopotential height anomalies associated with the stratospheric Asymmetric SAM (Figure \ref{fig:vertical-regression}a) are clearly constrained to the stratosphere, which underscores the uncoupling between the stratospheric and tropospheric Asymmetric SAM. The vertical structure of this signal tilts about 60\degree to the West between 100 hPa and 1 hPa, suggesting baroclinic processes. The signal in the stratosphere maximises near 10 hPa despite using the 50 hPa index for the regression.

The tropospheric Asymmetric SAM (Figure \ref{fig:vertical-regression}b) has significant signals that extend upwards to the uppermost levels considered. In the troposphere, the wave-3 structure is equivalent barotropic with maximum amplitude at roughly 250 hPa. The anomalies are larger in the Western hemisphere, where they extent into the stratosphere. In the Eastern hemisphere the wave-3 signal is smaller and confined to the troposphere while negative anomalies dominate in the stratosphere. So, while the tropospheric Asymmetric SAM index is associated with stratospheric geopotential anomalies, these do not project strongly onto the stratospheric Asymmetric SAM. The structures shown in Figure \ref{fig:vertical-regression}a Figure \ref{fig:vertical-regression}b in Figure \ref{fig:vertical-regression} are robust to the choice of index level. For any stratospheric (above 100 hPa) index, the resulting anomalies are very similar to the wave-1 structure with maximum near 10 hPa in Figure \ref{fig:vertical-regression}a. Conversely, for any tropospheric (below 100 hPa) index, the result is very similar to Figure \ref{fig:vertical-regression}b. The patterns mainly change in amplitude (not shown).

The wave-3 pattern from Figure \ref{fig:2d-regr}b.2 is very similar to the PSA Pattern \citep{mo1987, kidson1988} which is a teleconnection pattern associated with the ENSO \citep{karoly1989}. Indeed, \citet{fogt2011} showed that there is a significant relationship between the SAM and the ENSO. The correlation between the Full SAM and the ENSO as measured by the Oceanic Niño Index \citep[ONI,][]{bamston1997} is -0.16 (p-value \textless{} 0.001). This relationship is captured mainly by the Asymmetric SAM, as this index has a partial correlation of -0.26 (p-value \textless{} 0.001) with the ONI, consistent with \citet{fan2007}'s results, whereas the Symmetric SAM's partial correlation with the ONI is essentially null (0.019; p-value = 0.67). The same analysis was performed using the Multivariate ENSO Index \citep{wolter2011} and the Southern Oscillation Index \citep{ropelewski1987}, obtaining similar results. The latter allows to conclude that these results do not depend on the ENSO index used.

\hypertarget{impacts}{%
\subsection{Impacts}\label{impacts}}

\begin{figure*}
\includegraphics{regr-air-season-1} \caption{Regression of seasonal mean surface land air and sea temperature anomalies (Kelvin) with Asymmetric SAM and Symmetric SAM for the 1979 -- 2018 period. Black contours indicate areas with p-value smaller than 0.05 controlling for False Detection Rate. Gray areas in Antarctica are areas with have more than 15\% of missing data.}\label{fig:regr-air-season}
\end{figure*}

To assess the differences in the impacts associated with full, asymmetric and symmetric SAM, we regress s surface land air and sea temperature and precipitation onto each of the three SAM indices at 700 hPa. As shown in previous sections, the three indices are highly coherent in the troposphere, so we select this level to represent the tropospheric circulation for consistency with previous studies.

Figure \ref{fig:regr-air-season} shows regression coefficients of each index at 700 hPa with surface land air and sea temperature for each trimester. In Summer, positive values of the Full SAM index (Figure \ref{fig:regr-air-season}a.1) are associated with negative temperature anomalies near Antarctica which are surrounded by a ring of positive anomalies. The ring is not zonally symmetric, as there are four distinctive local maximums around 30\degree W, 120\degree W, 150\degree E and 90\degree E respectively. In the tropics, there are negative anomalies in the equatorial Pacific, consistent with the negative correlation between SAM and ENSO. Figure \ref{fig:regr-air-season}b.1 and c.1 show temperature anomalies associated with positive values of the Asymmetric and Symmetric SAM, respectively. Both the local maximums in the ring and the anomalies in the Pacific regions are present mostly on the Asymmetric SAM regression map, while temperature patterns linked to positive Symmetric SAM show a more zonally consistent ring and less relation to the tropics. Temperature anomalies in the Indian ocean, South Africa and Australia are strongly related to Asymmetric SAM. This signal is not present in the regression pattern with the Full SAM. Spring (Figure \ref{fig:regr-air-season}, row 4) features similar patterns but of smaller magnitude, with less regions where regressed anomalies have statistical significance.

In Autumn and Winter (rows 2 and 3 in Figure \ref{fig:regr-air-season}) the positive ring is only present through its local maximums in the regression with the Full SAM. There are also negative anomalies in Southern Australia, and positive anomalies over New Zealand and Southern South America. These patterns are not significant in the sense that there are no areas with p-values below 0.05 when controlling for FDR following \citet{wilks2016}. However, repeating this analysis with 2-meter air temperature from ERA5 resulted in similar patterns that were statistically significant (not shown). Moreover, similar features were observed in station measurements by \citet{jones2019}, although using data from 1957 to 2016.

The pattern of negative anomalies in the pole surrounded by positive anomalies roughly seen in all seasons -- although with varying intensity and small-scale details -- translates to enhanced meridional temperature gradient maximised in the zero line, which is consistent with the intensification and poleward migration of the westerlies commonly linked to the SAM through thermal wind balance. It's then not surprising to see it more clearly in association with the Symmetric SAM (at least in Summer and Spring).

Figure \ref{fig:regr-air-season} column b can be partially compared with Figure 11 from \citet{fogt2012}. Although they used station data from 1958 to 2001, main features are reproduced here, such as the strong signal in New Zealand and Australia in Summer and Spring.

Regression of the SAM indices with seasonal mean precipitation and 700 hPa geopotential height are shown in Figures \ref{fig:pp-regr-oceania} and \ref{fig:pp-regr-america} for Australasia and South America respectively. South Africa is not shown because no significant signal was detected there.

\begin{figure*}
\includegraphics{pp-regr-oceania-1} \caption{Regression of (row 1) annual and (rows 2 to 5) seasonal mean precipitation anomalies (mm per day, shading) and 700 hPa geopotential height anomalies (thin lines, positive values as solid lines and negative values as dashed lines) with (column a) Full SAM, (column (b) Asymmetric SAM and (column c) Symmetric SAM for the 1979 -- 2018 period. Thin lines are the Black contours indicate areas with p-value smaller than 0.05 controlling for False Detection Rate.}\label{fig:pp-regr-oceania}
\end{figure*}

In Australia, the annual regression shows that the Full SAM index is associated with positive precipitation anomalies in the southeastern region (Figure \ref{fig:pp-regr-oceania}a.1), in agreement with \citet{gillett2006}. The separation between Asymmetric and Symmetric SAM suggest that this positive anomaly is explained by the Symmetric SAM only in the East coast (Figure \ref{fig:pp-regr-oceania}c.1). Geopotential height anomalies associated with this index (black contours) are indicative of easterly flow from the Tasman Sea, which could explain the positive anomalies in precipitation as found by \citet{hendon2007}. The Asymmetric SAM appears related to positive precipitation anomalies in the West coast of Southeastern Australia (Figure \ref{fig:pp-regr-oceania}b.2), which could similarly be explained by the anomalous westerly circulation transporting moist air to the continent from the Indian Ocean.

The seasonal-based regressions show statistically significant anomalies only in Spring, when positive Full SAM is associated with positive precipitation anomalies in Eastern Australia (Figure \ref{fig:pp-regr-oceania}a.5). In this trimester, the Symmetric SAM seems to be associated with positive precipitation anomalies in a relatively reduced area of the East Coast (Figure \ref{fig:pp-regr-oceania}c.5) while the positive precipitation anomalies related with positive Asymmetric SAM affect all Eastern Australia (Figure \ref{fig:pp-regr-oceania}b.5).

In Summer, positive Full SAM index is associated with positive precipitation anomalies in Western and Eastern Australia, particularly in the North East (Figure \ref{fig:pp-regr-oceania}a.2). The Eastern part being dominated by the relationship with the Symmetric SAM and the Western, by the Asymmetric SAM. In Autumn, the regression with Full SAM shows positive anomalies in the North, similar to Summer, and a broad area of positive anomalies in the North-East to South-West direction. This structure seems to be associated with the Symmetric SAM, while the Northern positive values are associated with the Asymmetric SAM. In Winter we see the same NE to SW aligned anomaly (although with much reduced amplitude) that is also present only in relation with the Symmetric SAM. None of these regression coefficients are statistically significant at the 95\% level. The Spring signal is broadly consistent with \citet{hendon2007}, but whereas they also detected a strong signal in Summer, Figure \ref{fig:pp-regr-oceania}a.2 shows no statistically significant association (although the coefficients have the consistent sign).

\begin{figure*}
\includegraphics{pp-regr-america-1} \caption{Same as Figure \\ref{fig:pp-regr-oceania} but for South America.}\label{fig:pp-regr-america}
\end{figure*}

In South America (Figure \ref{fig:pp-regr-america}), regression using all seasons shows that positive SAM is associated with statistically significant negative precipitation anomalies in Southeastern South America (SESA) and Southern Chile and non-significant positive anomalies in South Brazil, near the South Atlantic Convergence Zone (SACZ) (Figure \ref{fig:pp-regr-america}a.1). Figure \ref{fig:pp-regr-america}b.1 and c.1 show that while the signal over SESA and southern Brazil is associated with the Asymmetric SAM, that in southern Chile is related to the Symmetric SAM.

Except Winter, seasonal-based regressions mirror this same pattern. Even if not statistically significant, they all show negative values in SESA and southern Chile along with positive values in southern Brazil in relation with the Full SAM. The separation of these features between the Asymmetric SAM and Symmetric SAM regression maps is also rather consistent.

The anomalous circulation at 700 hPa associated with the Symmetric SAM (panel c.1) indicate anomalous easterly flow over Southern Chile. This leads to reduced influx of moist air from the Pacific Ocean which, is the main source of precipitable water in that region \citep[e.g.][]{garreaud2007}. On the other hand, the anomalous circulation associated with positive values of Asymmetric SAM (panel b.1) in the Atlantic is anticyclonic in the South and cyclonic in the North. This promotes anomalous South-Easterly flow over SESA, which inhibits the flow of the South America Low-Level Jet to the region \citep{silvestri2009, zamboni2010}. This same pattern was found to be associated with increased precipitation in Southern Brazil during SACZ events \citep{rosso2018}. There is a small area of significative positive precipitation anomalies with SAM near central Argentina which is also present in the station-based analysis by \citet{gillett2006} that is explained by the Asymmetric SAM.

\hypertarget{conclusions}{%
\subsection{Conclusions}\label{conclusions}}

In this study we characterise the temporal and spatial variability of the zonally symmetric and asymmetric structure of the SAM. By projecting monthly geopotential fields at each level with the corresponding asymmetric and symmetric pattern, we created two indices for representing the zonally asymmetric and symmetric contributions of the SAM respectively.

The Asymmetric SAM index correlates strongly with the Symmetric SAM index. In the troposphere, this correlation is maximum at zero lag, while in the stratosphere is maximised with the Asymmetric SAM leading the Symmetric SAM by one month. Since most indices of the SAM are calculated using surface or near-surface conditions, this result would suggest that they might not be sensitive to the most dramatic changes in SAM variability.

The two-year periodicity we found in the stratospheric Symmetric SAM might point to a link between the SAM and the Quasi Biennial Oscillation. There is evidence of influence between the QBO and the Northern Annular Mode \citep[e.g.][]{holton1980, watson2014, zhang2020}, so it is not unlikely that the SAM would be similarly connected. However, establishing this link would require further research.
We observe a positive trend of SAM in Summer and Autumn, as was documented by previous studies {[}e.g. \citet{fogt2020}; and references therein{]} for low levels. We show that these trends maximise at 100 hPa, and are explained by the zonally symmetric component. We also find a statistically significant positive trend in the Symmetric component of the SAM in the stratosphere that is not evident in the Full SAM index. In contrast to \citet{fogt2012} who made the evaluation for the period 158 -- 2001, we find some evidence of the SAM becoming more zonally asymmetric, as there is a slight positive trend in the variance explained by the as the Asymmetric SAM explains an increasingly proportion of the total variance. This might be due either due to differneces in methodology or analysed period.

In the troposphere, the spatial patterns of geopotential associated with the Symmetric SAM are much closer to being fully annular than the patterns associated with the Full SAM index. The Asymmetric SAM, on the other hand, describes a wave-3 pattern with maximum amplitude in the Pacific This pattern extends in the troposphere but its maximum is located at 250 hPa, which also could suggest that surface-based indices are not optimum for capturing this variability. This wave-3 pattern is similar to the Pacific-South American Pattern, linked to ENSO variability. We found that the significant correlation that exists between the Full SAM index and the Oceanic Niño Index over this period, is captured entirely by the Asymmetric SAM index. This suggests that ENSO is linked to SAM exclusively through the variability in the latter's asymmetric component and thus, the Asymmetric SAM index could be a useful measure to further study that relationship.

Temperature anomalies associated with the Full SAM broadly show a pattern of negative anomalies at polar latitudes surrounded by positive anomalies, but with many deviations from symmetry. The Asymmetric SAM index explains a big portion of these deviations. In particular, the positive phase of the Asymmetric SAM is associated with colder temperatures over Southern Brazil, South Africa and Southern Australia, as well as the negative anomalies in the equatorial Pacific consistent with the ENSO-SAM relationship. These negative anomalies are particularly clear in the DJF and SON trimesters, which include the months in which the ENSO teleconnection is more active \citep[e.g.][]{cai2020a}.

In Australia the Full SAM is associated with positive precipitation anomalies in South East and this is explained by the Symmetric SAM. However, the Asymmetric SAM is associated with a small area of positive precipitation anomalies in the Eastern Coast of West Australia, maybe due to advection of moist air from the Indian Ocean. In South America, precipitation anomalies associated with the Full SAM are negative both in Southern Chile and Southeastern South America, and positive in Southern Brazil. These features are cleanly separated between the Asymmetric and Symmetric components. The Symmetric SAM explains the negative anomalies in Southern Chile and the Asymmetric SAM, the negative-positive dipole between Southeastern South America and Southern Brazil. Individual seasons mostly follow this pattern.

\citet{silvestri2009} suggests that precipitation impacts linked to the SAM changed strongly before and after 1980. In particular, the negative relationship with precipitation in South America was absent in some areas and switched sign in others in the earlier period. The correlation between ENSO and SAM is similarly non-stationary, also changing sign before the 1980s \citep{fogt2006, clem2013}. Seeing as both the ENSO-SAM relationship and most of the precipitation impacts in South America are captured by the Asymmetric SAM, the results presented here are most likely period-dependent. The decadal variations of the Asymmetric SAM should be focus of future studies.

By separating the zonally symmetric and zonally symmetric SAM signals, we show that the asymmetric component of the SAM has its unique variability, trends and impacts. This is particularly important in the context of a changing climate, as the impact on the SAM of ozone recovery is modeled as highly zonally symmetric, while the impact of increased concentration of greenhose gases has also a zonally asymmetric component \citep{arblaster2006}.

\begin{acknowledgements}
NOAA Global Surface Temperature (NOAAGlobalTemp) data provided by the NOAA/OAR/ESRL PSL, Boulder, Colorado, USA, from their Web site at https://psl.noaa.gov/ 
\end{acknowledgements}

\hypertarget{data-statement}{%
\subsection{Data statement}\label{data-statement}}

All data used in this paper is freely available from their respective sources. ERA5 data can be obtained via the Copernicus Climate Data Store (\url{https://cds.climate.copernicus.eu/cdsapp\#!/dataset/reanalysis-era5-pressure-levels-monthly-means}). NOAAGlobalTemp and GPCC precipitation data can be obtained through the NOAA Physical Sicences Laboratory website (\url{https://psl.noaa.gov/data/gridded/data.noaaglobaltemp.html} and \url{https://psl.noaa.gov/data/gridded/data.gpcc.html}). The Oceanic Niño Index is avaiable via NOAA's Climate Precidction Center: \url{https://www.cpc.ncep.noaa.gov/products/analysis_monitoring/ensostuff/detrend.nino34.ascii.txt}

A version-controlled repository of the code used to create this analysis, including the code used to download the data can be found at \url{https://github.com/eliocamp/asymsam}.

\newpage

\appendix

\counterwithin{figure}{section}

\hypertarget{extra-figures}{%
\section{Extra figures}\label{extra-figures}}

\newpage

\begin{figure}[ht]
\includegraphics{A1-1} \caption{Lag-correlation between Asymmetric SAM and Symmetric SAM index at each level. Negative lags imply Symmetric SAM leading Asymmetric SAM and vice versa. For the 1979 -- 2018 period.}\label{fig:A1}
\end{figure}

\begin{figure}
\includegraphics{A2-1} \caption{Fourier spectrum of each timeseries computed as Fourier transform smoothed with modified Daniell smoothers with widths 3 and 5. The shading indicates de 95\% confidence area derived by fitting an autorregressive model and computing the spectrum for 5000 simulated samples from the fitted autoregressive model (95\% of the simulated sampels had an amplitude equal or lower). The light line indicates the theoretical expected amplitude from the autorregressive model. For the 1979 -- 2018 period.}\label{fig:A2}
\end{figure}

\begin{figure}
\includegraphics{A3-1} \caption{50 hPa Geopotetnial height zonal anomalies (meters) of composites of positive and negative SAM months selected using $\pm1$ standard deviation as threshhold for the 1979 -- 2018 period. Numbers in the column headers are pattern correlation between SAM+ and SAM- composites and number of monthly fields used to construct the composites.}\label{fig:A3}
\end{figure}

\begin{figure}
\includegraphics{A4-1} \caption{700 hPa Geopotetnial height zonal anomalies (meters) of composites of positive and negative SAM months selected using $\pm1$ standard deviation as threshhold for the 1979 -- 2018 period. Numbers in the column headers are pattern correlation between SAM+ and SAM- composites and number of monthly fields used to construct the composites.}\label{fig:A4}
\end{figure}

\begin{figure}
\includegraphics{A5-1} \caption{Regression coefficients of 50 hPa and 700 hPa geopotential height zonal anomalies (meters) onto the standarised timeseries of the leading EOF computed for each season independently for the 1979 -- 2018 period.}\label{fig:A5}
\end{figure}

\begin{figure}
\includegraphics{A6-1} \caption{Regression of 50 hPa and 700 hPa geopotential height zonal anomalies (meters) onto the standarised timeseries of the leading EOF computed for the periods 1979 -- 1998 and 1999 -- 2018. Pattern correlation between both fields is 0.86 for the 50 hPa fields and 0.76 for the 700 hPa fields.}\label{fig:A6}
\end{figure}

\newpage

\bibliographystyle{spbasic}
\bibliography{AsymSAM,packages}

\end{document}
