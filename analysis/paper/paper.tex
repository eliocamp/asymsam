%% Version 5.0, 2 January 2020
%
%%%%%%%%%%%%%%%%%%%%%%%%%%%%%%%%%%%%%%%%%%%%%%%%%%%%%%%%%%%%%%%%%%%%%%
% TemplateV5.tex --  LaTeX-based template for submissions to the
% American Meteorological Society
%
%%%%%%%%%%%%%%%%%%%%%%%%%%%%%%%%%%%%%%%%%%%%%%%%%%%%%%%%%%%%%%%%%%%%%
% PREAMBLE
%%%%%%%%%%%%%%%%%%%%%%%%%%%%%%%%%%%%%%%%%%%%%%%%%%%%%%%%%%%%%%%%%%%%%

%% Start with one of the following:
% DOUBLE-SPACED VERSION FOR SUBMISSION TO THE AMS
\documentclass[]{ametsocV5}


% TWO-COLUMN JOURNAL PAGE LAYOUT---FOR AUTHOR USE ONLY
% \documentclass[twocol]{ametsocV5}


% Enter packages here. If too many math alphabets are used,
% remove unnecessary packages or define hmmax and bmmax as necessary.

\newcommand{\hmmax}{0}
\newcommand{\bmmax}{0}
\usepackage{amsmath,amsfonts,amssymb,bm}
\usepackage{mathptmx}%{times}
\usepackage{newtxtext}
\usepackage{newtxmath}

\usepackage{gensymb}
\usepackage{subfig}
\usepackage[inline]{showlabels}

%%%%%%%%%%%%%%%%%%%%%%%%%%%%%%%%

%%% To be entered by author:

%% May use \\ to break lines in title:

\title{New method to describe the zonal symmetries and asymmetries of the
Southern Annular Mode}


\authors{Elio Campitelli
\correspondingauthor{Elio Campitelli,elio.campitelli@cima.fcen.uba.ar}
and Leandro Díaz
}

\affiliation{Universidad de Buenos Aires, Facultad de Ciencias Exactas y Naturales,
Departamento de Ciencias de la Atmósfera y los Océanos, Buenos Aires,
Argentina CONICET -- Universidad de Buenos Aires, Centro de
Investigaciones del Mar y la Atmósfera (CIMA), Buenos Aires, Argentina
CNRS -- IRD -- CONICET -- UBA, Instituto Franco‐Argentino para el
Estudio del Clima y sus Impactos (UMI 3351 IFAECI), Buenos Aires,
Argentina}

\extraauthor{Carolina Vera
}


%%%%%%%%%%%%%%%%%%%%%%%%%%%%%%%%%%%%%%%%%%%%%%%%%%%%%%%%%%%%%%%%%%%%%
% ABSTRACT
%
% Enter your abstract here
% Abstracts should not exceed 250 words in length!
%


\abstract{Enter the text of your abstract here. This is a sample American
Meteorological Society (AMS) \LaTeX~template. This document provides
authors with instructions on the use of the AMS \LaTeX~template. Authors
should refer to the file amspaper.tex to review the actual \LaTeX~code
used to create this document. The template.tex file should be modified
by authors for their own manuscript.}

\begin{document}

%% Necessary!
\maketitle

\bibliographystyle{ametsoc2014}
%%%%%%%%%%%%%%%%%%%%%%%%%%%%%%%%%%%%%%%%%%%%%%%%%%%%%%%%%%%%%%%%%%%%%
% SIGNIFICANCE STATEMENT/CAPSULE SUMMARY
%%%%%%%%%%%%%%%%%%%%%%%%%%%%%%%%%%%%%%%%%%%%%%%%%%%%%%%%%%%%%%%%%%%%%
%
% If you are including an optional significance statement for a journal article or a required capsule summary for BAMS
% (see www.ametsoc.org/ams/index.cfm/publications/authors/journal-and-bams-authors/formatting-and-manuscript-components for details),
% please apply the necessary command as shown below:
%
\statement
This is significant becasue I wrote it.



%%%%%%%%%%%%%%%%%%%%%%%%%%%%%%%%%%%%%%%%%%%%%%%%%%%%%%%%%%%%%%%%%%%%%
% MAIN BODY OF PAPER
%%%%%%%%%%%%%%%%%%%%%%%%%%%%%%%%%%%%%%%%%%%%%%%%%%%%%%%%%%%%%%%%%%%%%
%

\section{Introduction}

The Southern Annular Mode (SAM) is the main mode of variability in the
Southern Hemisphere extratropical circulation \citep{rogers1982} in
daily, monthly, and decadal timescales {[}\citet{baldwin2001a};
fogt2006{]} and exerts an important influence in weather conditions such
as temperature and precipitation anomalies and sea ice concentration
\citep{fogt2020}. Its positive phase is traditionally described as
anomalously low pressures over Antarctica surrounded by a ring of
anomalous high pressures in middle-to-high latitudes.

However, computed as the leading Empirical Orthogonal Function (EOF) of
Sea Level Pressure or low-level geopotential height, the SAM spatial
structure contains noticeable deviations from this zonally symmetric
description, particularly in the Pacific Ocean region. These zonal
asymmetries are not widely studied, but previous work suggest that they
strongly modulate the regional impacts of the SAM, going as far as
reversing its relationship between precipitation in South America
\citep{silvestri2009}. At the very least, the fact that the SAM is not
entirely zonally symmetric hinders our ability to reconstruct its
historical variability prior to the availabilty of dense observations in
the Southern Hemisphere \citep{jones2009}.

We are not aware of any previous work which quantifies the temporal
variability of the asymmetric component of the SAM with the exception of
\citet{fogt2012}. However, their methods based on composites of positive
and negative SAM events leads to some issues, such as spatial patterns
derived from as little as 4 cases and from imbalanced periods (for
example, 5 of the 7 cases in their DJF SAM+ composite are from later
than 1988, whereas all of the 8 years in their DJF SAM- composite are
from earlier than 1988). This is particularly important due to the
inhomogeneities in renalysis products prior to the sattelite era and the
possible change in the asymmetric structure of the SAM
\citep{silvestri2009}.

Our objective is, then, to systematically characterise the zonally
asymmetric component of the SAM variability by constructing two indices
which aim to capture exclusively the variability of the symmetric and
asymmetric component each. We then analyse their temporal variability,
trends and vertical coherence. We study the spatial patterns described
by the variability exclusive to each index. Finally, we investigate
theri relationship with temperature and recipitation anomalies.

\section{Methods}

\subsubsection{Data}

To describe the Southern Annular Mode and its variability we used
monthly geopotential height at 2.5\degree longitude by
2.5\degree latitude of horizontal resolution and 37 vertical isobaric
levels from ERA5 \citep{hersbach2020} for the period 1979 to 2018. We
restrict our analysis to the post-satellite era to avoid any confounding
factors arising from the introduction of satellite observations.

We describe the relationship between the SAM indices and temperature and
precipitation. We use temperature data from NOAA's Merged Land Ocean
Global Surface Temperature Analysis V4.0.1 \citep{vose2012, smith2008},
which blends land and ocean temperature analysis into a monthly global
grid 5\degree lagitude by 5\degree longitude. For precipitation, we use
monthly, 0.5\degree latitude by 0.5\degree longitude data from the
Global Precipitation Climatology Centre \citep{schneider2015}.

\subsubsection{Definition of indices}

Traditionally the Souther Annular Mode (SAM) is defined as de leading
Empiral Orthogonal Mode (EOF) of sea level pressure or geopotential
height at lower levels \citep{ho2012}. Following \citet{baldwin2001}, we
extend that definition vertically and use the term SAM to refer to the
the leading EOF of the monthly anomalies of geopotential field south of
20\degree S at each level. We performed EOFs by computing the Singular
Value Decomposition of the data matrix consisting in 481 rows and 4176
columns (144 points of longitude and 29 points of latitude). We weighted
the values by the square root of the cosine of latitude to account for
the non-equal area of each gridpoint \citep{chung1999}.

To separate between the zonally symmetric and asymmetric components of
the SAM, we computed the zonal mean and anomalies of the full SAM
spatial pattern, as shown in Figure~\ref{fig:method} for 700hPa. The
full spatial signal (\(\mathrm{EOF_1}(\lambda, \phi)\)) is the sum of
the zonally asymmetric (\(\mathrm{EOF_1^*}(\lambda, \phi)\)) and
symmetric (\([\mathrm{EOF_1}](\lambda, \phi)\)) components. We then
compute the ``Full SAM'', ``Asymmetric SAM'' and ``Symmetric SAM''
indices as the regression coefficients of the regression ofeach monthly
geopotential field on the respective patterns (weighting by the cosine
of latitude). The three indices are normalised by dividing them by the
standard deviation of the ``Full'' index at each level. As a result, the
magnitude between indices is comparable. However, only ``Full'' index
will have unit standard deviation per definition. From the regression,
we also use the explained variance of each pattern as a indication of
the degere of symmetry or asymmetry of each monthly field.

\begin{figure*}
\includegraphics{method-1} \caption[Spatial patterns of the first EOF of 700 hPa geopotential height]{Spatial patterns of the first EOF of 700 hPa geopotential height. (a) Full field, (b) zonally asymmetric component and (c) zonally symmetric component. Arbitrary units.}\label{fig:method}
\end{figure*}

Our method assumes linearity in the asymmetric component of the SAM.
That is, we assume that zonal symmetries associated with positive SAM
are oposite and equal to the ones associated with negatie SAM.
\citet{fogt2012}'s composites (their Figure 4) suggest that this might
not be entirely valid, although we argue that much of that apparent
non-linearity is due to the heterogenous nature of the selected years
for constructing the composites. Using our data (from 1979 to 2018),
seasonal composites of zonal anomalies of 700 hPa geopotential height
for SAM+ (Full SAM index greater than 1 standard deviation) and SAM-
(smaller than negative 1 standard deviation) show relatively high
pattern correlations all seasons and are visually very linear (Figure
A9). Therefore, we belive that our method is at the very least a
reasonable approximation of the phenomenon.

By computing a single EOF pattern using data for all months we are
assuming that the zonal anomalies of the SAM are the same in all
seasons. Geopotential zonal anomalies computed by projecting the first
EOF \emph{of each season} are very similar to each other (Figure A10)
and show pattern correlations between 0.65 (DJF with JJA) and 0.9
(between MAM and SON). Based on this, we believe that our initial
assumption is not unreasonable.

Finally, we assume that the zonally asymmetric pattern is stationary in
time. \citet{silvestri2009} suggest that this might not be the case
between 1958 and 2004 but the period we analyse is much shorter
(1979-2018) so it's unlikely that we could observe significant changes.
Moreover, zonal asymmetry of the spatial patterns for the two halves of
the period (1979 to 1998 and 1999 to 2018) show no systematic change
(Figure A11).

\subsubsection{Regressions}

We perform linear regression to quantify the association between the SAM
indices and other variables. Since the Asymmetric and Symmetric SAM
indices are significantly correlated with each other, to capture the
variability explained uniquely by each index we use one multiple linear
regression instead of two simple linear regressions. To obtain the
linear coefficients of a variable \(X\) (geopotential, temperature,
precipitation, etc\ldots{}) with the Asymmetric SAM (\(SAM_a\)) and
Symmetric SAM (\(SAM_s\)) we fit the ecuation

\[
X(\lambda, \phi, t) = \alpha(\lambda, \phi) SAM_a + \beta(\lambda, \phi) SAM_s + X_0(\lambda, \phi) +  \epsilon(\lambda, \phi, t)
\]

where \(\lambda\) and \(\phi\) are the longitude and latitude, \(t\) is
the time, \(\alpha\) and \(\beta\) are the linear coefficients, \(X_0\)
and \(\epsilon\) are the constant and error terms. From this equarion,
\(\alpha\) represents the (linear) association of \(X\) with the
variability of the Asymmetric SAM that is not explained by the
variability of the Symmetric SAM; in other words, it is proportional to
the partial correlation of \(X\) and the Asymmetric SAM, controlling for
the effect of the Symmetric SAM and viceversa for \(\beta\).

At 2.5\degree by 2.5\degree resolution, a single regression field is
composed of thousands of regressions. In such case, using naive p-values
to test for significance leads to misleading results
\citep{walker1914, katz1991}. While there are multiple proposed
solutions in the literature, \citet{wilks2016} suggests that adjunting
p-values by controlling for the False Discovery Rate
\citep{benjamini1995} is a simple and effective method to ameliorate
this issue. Therefore, p-values showed in regression fields are all
adjusted following \citet{benjamini1995}.

When performing a separate regression for each trimester (DJF, MAM, JJA,
SON) we first average the relevant variables to obtain a single value
for each year and each trimester.

\section{Results}

\subsection{Temporal evolution}

\begin{figure*}
\includegraphics{asymsam-timeseries-1} \caption[Time series for the Asymmetric SAM and Symmetric SAM indices at (a) 50 hPa and (b) 700 hPa]{Time series for the Asymmetric SAM and Symmetric SAM indices at (a) 50 hPa and (b) 700 hPa. To the left, probability density estimate of each index. Series are standarised by the standard deviation of the Full SAM at each level.}\label{fig:asymsam-timeseries}
\end{figure*}

The temporal evolution of the Assymmetric and Symmetric SAM was firstly
asssesed. Figure \ref{fig:asymsam-timeseries} shows the corresponding
time series for 700 hPa and 50 hPa and their corresponding density
estimates. We selected these two levels as representative of the
tropospheric and stratospheric variability respectively. As will be
shown later, both indices are highly coherent within each atmospheric
layer, therefore is reasonable to take one level as representative of
each layer.

Month-to-month variability is evident for both indices, with noisy
variations in the low frequency. At first glance the series can be
distinguished by their distributions. Compared to the stratospheric
indices, the stratospheric indices are much more long-tailed; that is,
extreme values (both negative and positive) abound. The Asymmetric
series have both more variability in the higher frequencies than the
Symmetric series.

The stratospheric Symmetric SAM varies strongly with a two-year period,
which can be seen by spectral analysis (Figure A3). This might suggests
a link between stratospheric SAM variability and QBO. There is a local
peak at 2 years in the periodigram of the tropospheric Symmetric SAM
also, although it's not statistically significant. In the troposphere
the most significant peak of variability is found in the Asymmetric
index at around 3.6 months.

From Figure \ref{fig:asymsam-timeseries} we can see that the Asymemtric
and Symmetric time series appear to be correlated. Moreover, looking at
the extremes in the stratosphere, the Symmetric serie appears to lag the
Asymmetric series (see, for example, the positive events on late 1987).
We show these correlations, across all the levels of the reanalysis for
zero and -1 lag (Asymmetric index leading the Symmetric index), in
Figure \ref{fig:cor-lev}. Zero-lag correlations between the Asymmetric
and Symmetric series are relatively constant throught the troposphere,
fluctiating between 0.39 and 0.45. One-month-lag correlations are
similarly constant but significantly reduced to around 0.17. In the
stratosphere, zero-lag correlations drop to a minimum of 0.21 at 20 hPa
and then it increases again monotonically with height up to the
uppermost level of the reanalysis (although results near the top of the
models are to be interpreted with care). At the same time, one-month-lag
correlations increase with height. As a consequence, statospheric
Symmetric index tend to precede corresponding Asymmetric index.

\begin{figure}
\includegraphics{cor-lev-1} \caption[Correlation between the Symmetric SAM and Asymmetric SAM index at each level for lag zero and lag -1 (Symmetric leads Asymmetric)]{Correlation between the Symmetric SAM and Asymmetric SAM index at each level for lag zero and lag -1 (Symmetric leads Asymmetric).}\label{fig:cor-lev}
\end{figure}

\begin{figure*}
\includegraphics{cross-correlation-1} \caption[Cross correlation between levels of the (a) Full SAM, (b) Asymmetric SAM, and (c) Symmetric SAM]{Cross correlation between levels of the (a) Full SAM, (b) Asymmetric SAM, and (c) Symmetric SAM.}\label{fig:cross-correlation}
\end{figure*}

Figure \ref{fig:cross-correlation}a shows (zero-lag) cross-correlation
across levels for the Full, Symmetric and Asymmetric SAM indices. For
the Full SAM (panel a), high values below 100 hPa reflect the vertical
(zero-lag) coherency throughout the troposfere. Above 100 hPa
correlation between levels falls off more rapidly, indicating less
coherent (zero-lag) variability. Therefore, there is a non negligible
correlation between the troposphere and the lower-to-middle
stratosphere. Examining panels b and c, we see that the Asymemtric and
Symmetric SAM share the same high level of coherency in the troposphere
but they differ in their stratospheric behaviour. Stratospheric
coherency is stronger for the Asymmetric SAM than the Symemtric SAM. The
stratospheric Symmetric SAM seems to connect more strongly to the
trosposphere than the Asymmetric SAM.

\begin{figure*}
\includegraphics{trends-1} \caption[Decadal trends at each level for annual (row 1) and seasonal values (rows 2 to 5) for the period 1979-2018 and for the (column a) Full SAM index, (column b) Asymmetric SAM index, and (column c) Symmetric SAM index]{Decadal trends at each level for annual (row 1) and seasonal values (rows 2 to 5) for the period 1979-2018 and for the (column a) Full SAM index, (column b) Asymmetric SAM index, and (column c) Symmetric SAM index. Shading indicates the 95\% confidence interval.}\label{fig:trends}
\end{figure*}

The trends for each of the indices (Full, Symmetric, Assymetric) were
evaluate for the whole period 1979-2018 at each level (Figure
\ref{fig:trends}) for the whole year and separed by trimesters. The Full
SAM index presents a statistically significant trend (panel a.1) that
extends throught the troposphere up to about 50 hPa and reaches its
maximum value at 100 hPa. The seasonal trends (rest of column a)
indicate that positive trends are present in Autumn and particularly in
in Summer, where the 100 hPa maximum is much more defined. Positive
trends have been documented by previous studies \citep[e.g.][ and
references therein]{fogt2020} using indices of the SAM based on surface
or near-surface circulation.

By separating the SAM signal in its Asymmetric and Symmetric parts, we
can not only see that these trends are almost entirely due to the
Symmetric component (colum b vs.~column c), but in some cases the trends
become more clear. In Summer, the Asymmetric SAM has a statistically non
significant negative trend in the middle troposphere that obscures the
trend in the Full index; as a result, trends computed using only the
Symmetric component are more clear (compare the shading region in panel
a.2 and c.2). In Autumn, using the Symmetric SAM reveals a statistically
significant positive trend in the stratosphere that is not significant
using the Full index.

We stress that these are only linear trends during the whole period and
the absence of a statistically significant signal should not be taken as
evidence of no sistematic change. In particular, going back to Figure
\ref{fig:asymsam-timeseries}, we can see an evident change in the
stratospheric Asymemtric component (red line in panel a) between the
90's, when we see a dominance of extreme negative values, and the 00's,
when we see the inverse. This change is restricted to the Winter months:
the linear trend for JJA starting in 1990 for the Asymmetric component
at 50hPa is \(0.37 \pm 0.22\).

Figure \ref{fig:r-squared-trend} shows decadal trends for the explained
variance of each index. There is no evidencie of a significant trend in
the stratosphere. In the troposphere, there is a positive trend for the
Asymmetric SAM and no significant trend for the Symmetric SAM. This
suggest that the SAM has become more asymmetric in the period from 1979
to 2018. The change is slight, though; of the order of 1\% icreased
explained variance per decade.

\begin{figure}
\includegraphics{r-squared-trend-1} \caption[Decadal trends of the variance explained by the Asymmetric and Symmetric SAM at each level for the period 1979-2018]{Decadal trends of the variance explained by the Asymmetric and Symmetric SAM at each level for the period 1979-2018. Shading indicates the 95\% confidence interval.}\label{fig:r-squared-trend}
\end{figure}

\subsection{Spatial patterns}

\begin{figure*}
\includegraphics{2d-regr-1} \caption[Regression patterns of geopotential height (meters) at (row 1) 50 hPa and (row 2) 700 hPa with the (column a) Full SAM, (column b) Asymmetric SAM, and (column c) Symmetric SAM]{Regression patterns of geopotential height (meters) at (row 1) 50 hPa and (row 2) 700 hPa with the (column a) Full SAM, (column b) Asymmetric SAM, and (column c) Symmetric SAM. The regression patterns for Asymmetric and Symmetric SAM are the result of one multiple regression using both indices, not of two simple regressions involving each index by itsef. Points marked on panel b.2 are the location of the reference points used by \cite{raphael2004} for their Zonal Wave 3 index. }\label{fig:2d-regr}
\end{figure*}

To show if, and to what extent, the Asymmetric and Symmetric SAM
inidices indeed capture the asymmetric and symmetric compoment of the
SAM respectively, we computed the spatial regression of geopotential
height anomalies on these indices and the Full SAM index. Figure
\ref{fig:2d-regr} shows these regressions. Regression coefficients in
column a are computed using the Full SAM. Regression coefficients in
columns b and c are computed using multiple regression using the
Asymmetric and Symmetric indices at the same time. Thus, they are to be
interpreted as the patterns associated with each index, removing the
variability (linearly) explained by the other index.

In the stratosphere, the spatial pattern associated with the Full SAM is
more clearly dominated by a zonally symmetric, monopolar structure
(panel a.1) which is, however, not perfectly centered in the South Pole.
The monopole obtained by multiple regression with the Asymmetric and
Symmetric SAM (panel c.1) is much more symmetric and the shift from
total symmetry is captured by the regression pattern of the Asymmetric
SAM as a wave-1 with maximum anomalies above the Belinghausen Sea on the
Western Hemisphere and and Davids Sea in the Eastern Hemisphere (panel
b.1).

In the troposphere, panel a.2 shows the well known combination of
zonally symmetrical annular mode with zonal asymmetries in the form of a
wave-3. The regression using the Asymmetric and Symmetric SAM indices
successfully disentangle both structures. The Asymmetric component gives
rise to a cleaner zonal wave (panel b.2) and the Symemtric component is
assosiated with an trully annular mode, almost devoid of zonal
asymmetries (panel c.2). The wave-3 pattern observed in panel b.2 is
rotated by half a wavelength from the average position of the mean
wave-3 pattern asociated with \citet{raphael2004}'s ZW3 index, whose
reference locations are marked with points in the figure. Thus, the
tropospheric Asymmetric SAM index represents a zonal displacement in the
position of the climatological wave-3 pattern.

\begin{figure}
\includegraphics{wave-amplitude-1} \caption{Amplitude (meters) of zonal waves of the geopotential height regression patterns in Figure \ref{fig:2d-regr} for zonal waves with wave-number 0, 1, 2, and 3, where wave-number 0 represents the amplitude of the zonal mean. Note the differnet x axis.}\label{fig:wave-amplitude}
\end{figure}

The amplitude of first zonal wave numbers at each latitude at 50 hPa and
700 hPa is shown in Figure \ref{fig:wave-amplitude}, where wave number
zero represents the amplitude of the zonal mean. Column b shows that the
Asymmetric SAM is overwhelmingly dominated by wave 1 in the stratosphere
(panel b), while in the troposphere it is composed of zonal waves 3 to 1
in decreasing level of importance (panel b). Looking at panel b.2 from
Figure \ref{fig:2d-regr}, it becomes apparent that zonal wavess 1 and 2
modulate the amplitude of zonal wave 3, which --as mentioned before-- is
larger in the Western Hemisphere than in the Easten Hemisphere.

\begin{figure}
\includegraphics{vertical-regression-1} \caption[Regression between monthly geopotential anomalies (meters) averaged betweeen 65\degree and 40\degree S and the Asymmetric SAM index (extracted from multiple regression including the Symmetric SAM)]{Regression between monthly geopotential anomalies (meters) averaged betweeen 65\degree and 40\degree S and the Asymmetric SAM index (extracted from multiple regression including the Symmetric SAM). (a) With the Asymmetric SAM in 50 hPa and (b) in 700 hPa.}\label{fig:vertical-regression}
\end{figure}

To analyse the vertical structure of the geopotential anomalies
asociated with the asymetric SAM index, we show a vertical cross section
of regressions of mean geopotential height between 65\degree S and
40\degree S for the 50 hPa Asymmetric SAM index (panel a) and for the
700 hPa Asymmetric SAM index (panel b) (Figure
\ref{fig:vertical-regression}). The geopotential anomalies associated
with the stratospheric Asymmetric SAM (panel a) are clearly constrained
to the stratosphere, which underscores the uncoupling between the
stratospheric and tropospheric Asymmetric SAM. The vertical structure of
this signal tilts about 60\degree to the West between 100 hPa and 1 hPa,
suggesting baroclinic processes. Interestingly, the signal in the
stratosphere maximises near 10 hPa despite using the 50 hPa index for
the regression.

The tropospheric Asymmetric SAM (panel b) has significant signals that
extend upwards to the uppermost levels of the reanalysis. In the
troposphere, the wave-3 structure is equivalent barotropic with maximum
amplitude at roughly 250 hPa. The anomalies are much more intense in the
Western hemisphere, where they extent into the stratosphere. In the
Eastern hemisphere the wave-3 signal is weaker and confined to the
troosphere while negative anomalies dominate in the stratosphere. So,
while the tropospheric Asymmetric SAM index is associated with
stratospheric geopotential anomalies, these do not project strongly onto
the stratospheric Asymmetric SAM.

The structures shown in panels a and b in Figure
\ref{fig:vertical-regression} are surprisignly robust to the choice of
index level. For any stratospheric (above 100 hPa) index, the resulting
anomalies are very similar to the wave-1 structure with maximum near 10
hPa in panel a. Conversely, for any tropospheric (below 100 hPa) index,
the result is very similar to panel b. The patterns mainly change in
amplitude.

The wave-3 pattern from Figure \ref{fig:2d-regr} panel b.2 is very
similar to the Pacific-South American Pattern \citep{mo1987, kidson1988}
which is a teleconnection pattern associated with the ENSO
\citep{karoly1989}. Indeed, \citet{fogt2011} showed that there is a
significant relationship between the SAM and the ENSO. The correlation
between the full SAM and the ENSO as measured by the Oceanic Niño Index
\citep{bamston1997} (ONI) is -0.16. Consistent with \citet{fan2007}, we
show that this relationship is captured entirely the Asymmetric SAM, as
this index has a partial correlation of -0.26 with the ONI controlling
for the effect of the Symmetric SAM, whereas the Symmetric SAM's partial
correlation with the ONI is essentially null (0.019). We performed the
same analysis using the Multivariate Enso Index \citep{wolter2011} and
the Southern Oscillation Index \citep{ropelewski1987} to conclude that
these results do not depend on the ENSO index used.

\subsection{Impacts}

\begin{figure*}
\includegraphics{regr-air-season-1} \caption[Regression pattern of season mean surface temperature (Kelvin) with Asymmetric SAM and Symmetric SAM]{Regression pattern of season mean surface temperature (Kelvin) with Asymmetric SAM and Symmetric SAM. Black contours indicate areas with p-value smaller than 0.05 controlling for False Detection Rate. Gray areas in Antarctica are areas with have more than 15\% of missing data.}\label{fig:regr-air-season}
\end{figure*}

The SAM has been shown to be associated with important surface variables
such as temperature and precipitacion \citep[e.g.][and see
\citet{fogt2020} for a review]{gillett2006}. Naturally, most studies on
the surface impacts of the SAM are based on an index identical or
analogous to what we call Full SAM index (\citet{fogt2012} being the
only exception that we are aware of). We regress surface temperature and
precipitation onto each of the three SAM indices to see if there are
diferent surface impact associated with the asymmetric and symmetric SAM
circulation.

Figure \ref{fig:regr-air-season} shows regression coefficients of each
index at 700 hPa with surface temperature for each trimester. In Summer
positive values of the Full SAM index (panel a.1) are associated with
negative temperature anomalies near Antarctica which are surrounded by a
ring of positive anomalies. The ring is not zonally symmetric, as there
are three clear local maximums around 30\degree W, 15\degree E and
50\degree E and a local minimum (with negative sign) around
120\degree W. In the tropics, there are negative anomalies in the
equatorial Pacific, consistent with the negative correlation between SAM
and ENSO. Panels b.1 and c.1 show temperature anomalies associated with
positive values of the Asymmetric and Symmetric SAM, respectively. Both
the local maximums in the ring and the anomalies in the Pacific regions
are present mostly on the Asymetric SAM regression map, while
temperature patterns linked to positive Symmetric SAM show a more
zonally consistent ring and less relation to the tropics. Noticeable,
temperature anomalies in the Indian ocean, South Africa and Australia
are strongly related to positive values of Asymmetric SAM. This signal
is not present in the regression pattern with the Full SAM. Spring (row
4) features very similar patterns but of generally smaller in magnitude
and statistical significance.

In Autumn and Winter (rows 2 and 3) the postive ring is only present
through its local maximums in the regression with the Full SAM. There
are also negative anomalies in Southern Australia, and positive
anomalies over New Zealand and Southern South America. These patterns
are not significant in the sense that there are no areas with p-values
below 0.05 when controlling for FDR following \citet{wilks2016}.
However, repeating this analysis with 2-meter temperature from ERA5
resulted in similar patterns that were statistically significant.
Moreover, similar features were observed in station measurements by
\citet{jones2019}, although using data from 1957 to 2016.

The pattern of negative anomalies in the pole surrounded by positive
anomalies roughly seen in all seasons --although with varying intensity
and small-scale details-- is consistent with the intensification and
poleward migration of the westerlies commonly linked to the SAM. It's
then not surprising to see it more clearly in association with the
Symemtric SAM (at least in Summer and Spring).

These results suggests that Asymmetric and Symmetric SAM indices are
associated with overall distinct temperature patterns which may not be
apparent when using the Full SAM index.

Figure \ref{fig:regr-air-season} column b can be partially compared with
Figure 11 from \citet{fogt2012}. Although they used station data from
1958 to 2001, a lot of the characteristics are reproduced here, such as
the strong signal in New Zealand and Australia in Summer and Spring.

Regression of the SAM indicies with seasonal mean precipitation and 700
hPa geopotential height are shown Figures \ref{fig:pp-regr-oceania} and
\ref{fig:pp-regr-america} for Australasia and South America
respectively. South Africa is not shown because no significant signal
was detected there.

\begin{figure*}
\includegraphics{pp-regr-oceania-1} \caption[Regression pattern of (row 1) annual and (rows 2 to 5) season mean precipitation anomalies (mm per day, shading) and 700 hPa geopotential height (thin lines, positive values as solid lines and negative values as dashed lines) with (column a) Full SAM, (column (b) Asymmetric SAM and (column c) Symmetric SAM]{Regression pattern of (row 1) annual and (rows 2 to 5) season mean precipitation anomalies (mm per day, shading) and 700 hPa geopotential height (thin lines, positive values as solid lines and negative values as dashed lines) with (column a) Full SAM, (column (b) Asymmetric SAM and (column c) Symmetric SAM. thin lines are the Black contours indicate areas with p-value smaller than 0.05 controlling for False Detection Rate.}\label{fig:pp-regr-oceania}
\end{figure*}

In Australia, the annual regression shows that the Full SAM index is
positively associated with precipitation in the Southeastern region
(Figure \ref{fig:pp-regr-oceania} panel a.1), which reproduces the
results from \citet{gillett2006}. The separation between Asymmetric and
Symmetric SAM suggest that this positive anomaly is explained by the
Symmetric SAM only in the East coast (panel c.1). Geopotential anomalies
associated with this index (black contours) are indicative of easterly
flow from the Tasman Sea, which could explain the positive anomalies in
precipitation as found by \citet{hendon2007}. The Asymmetric SAM appears
related to increased precipitation in the West coast of Southeastern
Australia (panel b.2), which could similarly be explained by the
anomalous westerly circulation transporting moist air to the continent
from the Indian Ocean.

This Spring signal is broadly consistent with \citet{hendon2007}, but
whereas \citet{hendon2007} also detected a strong signal in Summer,
panel a.2 shows no statistically significant association (although the
coeffcients have the consistent sign).

The seasonal-level regressions show statistically significant anomalies
only in Spring, when positive Full SAM is associated with positive
precipitation anomalies in Eastern Australia (panel a.5). In this
trimester the Symmetric SAM seems to be associated with precipitation in
a relatively reduced area of the East Coast (panel c.5) while the
positive precipitation anomalies related with positive Asymmetric SAM
affect all Estern Australia (panel b.5).

In Summer, positive Full SAM index is associated with with positive
precipitation anomalies in Western and Eastern Australia, particularly
in the North East (panel a.2). The Eastern part being dominated by the
relationship with the Symmetric SAM and the Western, by the Asymemtric
SAM. In Autumn, the regression with Full SAM shows positive values in
the North, similar to Summer, and a broad area of positive values in the
North-East to South-West direction. This structure seems to be
associated with the Symmetric SAM, while the Northern positive values
are associated with the Asymmetric SAM. In Winter we see the same NE to
SW aligned anomaly (although with much reduced amplitude) that is also
present only in relation with the Symmetric SAM. None of these
regression coefficients are statistically signifincat at the 95\% level

\begin{figure*}
\includegraphics{pp-regr-america-1} \caption{Same as Figure \ref{fig:pp-regr-oceania} but for South America.}\label{fig:pp-regr-america}
\end{figure*}

In South America (Figure \ref{fig:pp-regr-america}), the annual-level
regression shows that the SAM is associated with statistically
significant precipitation decrease in Southeastern South America (SESA)
and Southern Chile and non-significant increase in South Brazil, near
the South Atlantic Convergence Zone (SACZ) (panel a.1). Panels b.1 and
c.1 show a remarkably clean separation between the Asymmeric SAM
--associated with the Southeastern South American and Southern Brazilian
signals-- and the Symmetric SAM --associated with the signal in Southern
Chile.

Except Winter, seasonal-level regressions mirror this same pattern. Even
if not statistically significant, they all show negative values in
Southeastern South America and Southern Chile along with positive values
in Southern Brazil in relation with the Full SAM. The separation of
these features between the Asymmetric SAM and Symmetric SAM regression
maps is also rather consistent.

The anomalous circulation at 700 hPa associated with the Symmetric SAM
(panel c.1) indicate anomalous Easterly flow over Southern Chile. This
leads to reduced influx of moist air from the Pacific Ocean which, is
the main source of precipitable water in that region. On the other hand,
the anomalous circulation associated with positive values of Asymmetric
SAM (panel b.1) in the Atlantic is anticyclonic in the South and
cyclonic in the North. This creates anomalous South-Easterly flow over
Southeastern South America, which inhibits the flow of the Low Level Jet
to the region \citep[\citet{zamboni2010}]{silvestri2009}. This same
pattern was found to be associated with increased precipitation in
Southern Brazil during South Atlantic Convergence Zone events
\citep{rosso2018}.

There is a small area of increased precipitation with SAM near central
Argentina which is also present in the station-based analysis by
\citet{gillett2006} and that is explained by the Asymmmetric SAM.

\subsection{Conclusions}

In this study we tried to systematically characterise the variability of
the zonally symmetric and zonally asymmetric structure of the SAM. By
projecting monthly geopotential fields at each level with the
corresponding asymmetric and symmetric pattern, we created two indices
representing the zonally asymmetric and zonally symmetric contributions
of the SAM respectively.

As expected, the Asymmetric SAM index correlates strongly with the
Symmetric SAM index. In the troposphere, this correlation is maximum at
zero lag, while in the stratosphere is maximised with the Asymmetric SAM
leading the Symmetric SAM by one month. Since most indices of the SAM
are calculated using suftace or near-surface conditions, this result
would suggest that they might not be sensitive to the most dramatic
changes in SAM variability.

The two-year periodcty we found in the stratospheric Symmetric SAM might
point to a link between the SAM and the Quasi Biennial Oscillation.
There is evidence of influence between the QBO and the Northern Annular
Mode \citep[e.g.][\citet{watson2014}, \citet{zhang2020}]{holton1980}, so
it's not unlikely that the SAM would be similarly connected. However
establishing this link would require further research.

As documented by previous studies, such as \citet{fogt2020} (and
references therein), we observe a positive trend towards positive SAM in
Summer and Autumn. We show that these trends are maximised at the 100
hPa level and are explained by the zonally symmetric component. We also
find a statistically significant positive trend in the Symmetric
component of the SAM in the stratosphere that is not apparent in the
Full SAM index. In contrast to \citet{fogt2012} we find some evidence of
the SAM becoming more zonally asymmetric, as there is a slight positive
trend in the variance explained by the as the Asymmetric SAM explains an
increasingly proportion of the total variance.

In the troposphere, the spatial patterns of geopotential associated with
the Symmetric SAM is much closer to being trully annular than the
patterns associated with the Full SAM index. The Asymmetric SAM, on the
other hand, describes a wave-3 pattern with maximum amplitude in the
Pacific region and whose phae is rotated a quarter wavelength from the
mean zonal wave 3 described by \citet{raphael2004}'s index. This pattern
extends in the troposphere but its maximum is located at 250 hPa, which
also could suggest that surface-based indices are not optimum for
capturing this variability.

This wave-3 pattern is similar to the Pacific-South American Pattern,
which is a teleconnection pattern linked to ENSO variability. We found
that the significant correlation that exists between the Full SAM index
and the Oceanic Niño Index is captured entirely by the Asymmetric SAM
index. This suggests that ENSO is linked to SAM exclusively through the
variability in the latter's Asymmetric compoment.

Temperature anomalies associated with the Full SAM broadly show a
pattern of negative anomalies at polar latitudes surrounded by positive
anomalies, but with many deviations from symmetry. The Asymemtric SAM
index explains a big portion of these deviations. In particular, the
positive phase of the Asymmetric SAM is associated with colder
temperatures over Southern Brazil, South Africa and Southern Australia,
as well as the negative anomalies in the equatorial Pacific consistent
with the ENSO-SAM relationship delineated above. These are particularly
clear in the DJF and SON trimesters, which include the months in which
the ENSO teleconnection is more active
\citep{cazes-boezio2003, fogt2011, cai2020a}.

In Australia the Full SAM is associated with positive precipitation
anomalies in South East and this is explained by the Symemtric SAM.
However, the Asymmetric SAM is associated with a small area of positive
precipitation anomalies in the Eastern Coast of West Australia, maybe
due to advection of moist air from the Indian Ocean.

In South America, precipitation anomalies associated with the Full SAM
are negative both in Southern Chile and Southeastern South America, and
positive in Southern Brazil. This features are cleanly separated between
the Asymmetric and Symmetric components. The Symmetric SAM explains the
negative anomalies in Southern Chile and the Asymmetric SAM, the
negative-positive dipole between Southeastern South America and Southern
Brazil. Individual seasons mostly follow this pattern.

\citet{silvestri2009} suggests that precipitation impacts linked to the
SAM changed rather dramatically before and after 1980. In particular,
the negative relationship with precipitation in South America was absent
in some areas and switched sign in other in the earlier period. The
correlation between ENSO and SAM is similarly non-stationary, also
disapearing before 1973.

Seeing as both the ENSO-SAM relationship and most of the precipitation
imacts in South America are captured by the Asymmetric SAM, the results
presented here are most likely period-dependent. Therefore, is very
likely that if we were to repeat this analysis using pre-satellite data,
the resulting Asymmetric SAM would look very different.

\acknowledgments

NOAA Global Surface Temperature (NOAAGlobalTemp) data provided by the
NOAA/OAR/ESRL PSL, Boulder, Colorado, USA, from their Web site at
https://psl.noaa.gov/

\bibliography{AsymSAM}

\newpage

\appendix

\appendixtitle{Extra figures}

\begin{figure}
\includegraphics{A1-1} \appendcaption{A1}{Lag-correlation between Asymmetric SAM and Symmetric SAM index at each level. Negative lags imply Symmetric SAM leading Asymmetric SAM and vice versa.}\label{fig:A1}
\end{figure}

\begin{figure}
\includegraphics{A3-1} \appendcaption{A3}{Fourier spectrum of each timeseries computed as Fourier transform smoothed with modified Daniell smoothers with withs 3 and 5. The shading indicates de 95\% confidence area derived by fitting an autorregressive model and computing the spectrum for 5000 simulated samples from the fitted autoregressive model (95\% of the simulated sampels had an amplitude equal or lower). The light line indicates the theoretical expected amplitude from the autorregressive model.}\label{fig:A3}
\end{figure}

\begin{figure}
\includegraphics{A9-1} \caption[700 hPa Geopotetnial height zonal anomalies (meters) of composites of positive and negative SAM months selected using $\pm1$ standard deviation as threshhold]{700 hPa Geopotetnial height zonal anomalies (meters) of composites of positive and negative SAM months selected using $\pm1$ standard deviation as threshhold. Numbers in the column headers are pattern correlation between SAM+ and SAM- composites and number of monthly fields used to construct the composites.}\label{fig:A9}
\end{figure}

\begin{figure}
\includegraphics{A10-1} \caption[Regression of 700 hPa geopotential height zonal anomalies (meters) onto the standarised timeseries of the leading EOF computed for each season independently]{Regression of 700 hPa geopotential height zonal anomalies (meters) onto the standarised timeseries of the leading EOF computed for each season independently.}\label{fig:A10}
\end{figure}

\begin{figure}
\includegraphics{A11-1} \caption[Regression of 700 hPa geopotential height zonal anomalies (meters) onto the standarised timeseries of the leading EOF computed for the periods 1979 to 1998 and 1999 to 2018]{Regression of 700 hPa geopotential height zonal anomalies (meters) onto the standarised timeseries of the leading EOF computed for the periods 1979 to 1998 and 1999 to 2018. Pattern correlation between both fields is 0.76.}\label{fig:A11}
\end{figure}


\end{document}
