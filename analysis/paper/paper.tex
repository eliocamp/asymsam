%% Version 5.0, 2 January 2020
%
%%%%%%%%%%%%%%%%%%%%%%%%%%%%%%%%%%%%%%%%%%%%%%%%%%%%%%%%%%%%%%%%%%%%%%
% TemplateV5.tex --  LaTeX-based template for submissions to the
% American Meteorological Society
%
%%%%%%%%%%%%%%%%%%%%%%%%%%%%%%%%%%%%%%%%%%%%%%%%%%%%%%%%%%%%%%%%%%%%%
% PREAMBLE
%%%%%%%%%%%%%%%%%%%%%%%%%%%%%%%%%%%%%%%%%%%%%%%%%%%%%%%%%%%%%%%%%%%%%

%% Start with one of the following:
% DOUBLE-SPACED VERSION FOR SUBMISSION TO THE AMS
\documentclass[]{ametsocV5}


% TWO-COLUMN JOURNAL PAGE LAYOUT---FOR AUTHOR USE ONLY
% \documentclass[twocol]{ametsocV5}


% Enter packages here. If too many math alphabets are used,
% remove unnecessary packages or define hmmax and bmmax as necessary.

\newcommand{\hmmax}{0}
\newcommand{\bmmax}{0}
\usepackage{amsmath,amsfonts,amssymb,bm}
\usepackage{mathptmx}%{times}
\usepackage{newtxtext}
\usepackage{newtxmath}

\usepackage{gensymb}

%%%%%%%%%%%%%%%%%%%%%%%%%%%%%%%%

%%% To be entered by author:

%% May use \\ to break lines in title:

\title{Title here}


\authors{Elio Campitelli
\correspondingauthor{Elio Campitelli,elio.campitelli@cima.fcen.uba.ar}
and Leandro Díaz
}

\affiliation{CIMA UBA blablabla}

\extraauthor{Carolina Vera
}


%%%%%%%%%%%%%%%%%%%%%%%%%%%%%%%%%%%%%%%%%%%%%%%%%%%%%%%%%%%%%%%%%%%%%
% ABSTRACT
%
% Enter your abstract here
% Abstracts should not exceed 250 words in length!
%


\abstract{Enter the text of your abstract here. This is a sample American
Meteorological Society (AMS) \LaTeX~template. This document provides
authors with instructions on the use of the AMS \LaTeX~template. Authors
should refer to the file amspaper.tex to review the actual \LaTeX~code
used to create this document. The template.tex file should be modified
by authors for their own manuscript.}

\begin{document}

%% Necessary!
\maketitle

\bibliographystyle{ametsoc2014}
%%%%%%%%%%%%%%%%%%%%%%%%%%%%%%%%%%%%%%%%%%%%%%%%%%%%%%%%%%%%%%%%%%%%%
% SIGNIFICANCE STATEMENT/CAPSULE SUMMARY
%%%%%%%%%%%%%%%%%%%%%%%%%%%%%%%%%%%%%%%%%%%%%%%%%%%%%%%%%%%%%%%%%%%%%
%
% If you are including an optional significance statement for a journal article or a required capsule summary for BAMS
% (see www.ametsoc.org/ams/index.cfm/publications/authors/journal-and-bams-authors/formatting-and-manuscript-components for details),
% please apply the necessary command as shown below:
%
\statement
This is significant becasue I wrote it.



%%%%%%%%%%%%%%%%%%%%%%%%%%%%%%%%%%%%%%%%%%%%%%%%%%%%%%%%%%%%%%%%%%%%%
% MAIN BODY OF PAPER
%%%%%%%%%%%%%%%%%%%%%%%%%%%%%%%%%%%%%%%%%%%%%%%%%%%%%%%%%%%%%%%%%%%%%
%

\section{Introduction}

yada yada SAM yada yada circulation.. yada yada so important. yada yada
many impacts.

\section{Methods}

\subsubsection{Data}

We used monthly geopotential height at 2.5 longitude by 2.5 latitude
resolution from ERA5 \citep{hersbach} for the period 1979 to 2018
(inclusive).

Monthly temperature NOAA Global Surface Temperature (NOAAGlobalTemp) 5.0
degree latitude x 5.0 degree longitude global grid
\citep{vose2012, smith2008}. The same analysis was carried out using
CRUTEM4 \citep{osborn2014} (not shown).

We used monthly precipitation data from CPC Merged Analysis of
Precipitation \citep{xie1997} 2.5 degree latitude x 2.5 degree
longitude.

\subsubsection{Definition of indices}

We defined the Southern Annular Mode (SAM) as the leading EOF of the
monthly anomalies of geopotential field at 700 hPa south of 20\degree S
(citation?). The EOF was performed by computing the Singular Value
Decomposition of the data matrix consisting in 481 rows and 4176 columns
(144 points of longitude and 29 points of latitude). The values where
weighted by the square root of the cosine of latitude to account for the
non-equal area of each gridpoint \citep{chung1999}. This same method was
used at the rest of the levels considered in this paper.

To separate between the zonally symmetric and asymmetric components of
the SAM, we computed the zonal mean and anomalies of the full SAM
spatial pattern. The results are shown in Figure~\ref{fig:method} for
700hPa. The full spatial signal (\(\mathrm{EOF_1}(\lambda, \phi)\)) is
the sum of the zonally asymmetric (\(\mathrm{EOF_1^*}(\lambda, \phi)\))
and symmetric (\([\mathrm{EOF_1}](\lambda, \phi)\)) components. We then
compute the ``Full'', ``Asymmetric'' and ``Symmetric'' indices, by
regressing each geopotential field on these patterns (weighting by the
cosine of latitude).

The three indices are normalised by dividing them by the standard
deviation of the ``Full'' index at each level. This means that comparing
the magnitude between indices is meaningful, but it also means that not
every index will have unit standard deviation.

\begin{figure*}
\includegraphics{method-1} \caption[Spatial patterns of the first EOF of 700 hPa geopotential height]{Spatial patterns of the first EOF of 700 hPa geopotential height. Full field (left), zonally asymmetric component (middle) and zonally symmetric component (right). Arbitrary units.}\label{fig:method}
\end{figure*}

\subsubsection{Significance}

We adjusted p-values for False Detection Rate following
\citet{wilks2016}.

\section{Results}

\subsection{Temporal evolution}

\begin{figure*}
\includegraphics{asymsam-timeseries-1} \caption[Time series for the asymmetric SAM and symmetric SAM]{Time series for the asymmetric SAM and symmetric SAM.}\label{fig:asymsam-timeseries}
\end{figure*}

Figure \ref{fig:asymsam-timeseries} shows the resulting Asymmetric and
Symmetric time series corresponding to 700 and 30hPa. blablababla
\#FIXME

\begin{itemize}
\item
  stratosphere clearly nor normally distributed. a lot of values near 0
  and some relatively high outliers. Especially true int he case of the
  asymmetric index. High frequency variablity.
\item
  In both levels, there's correlation between the series (expected),
\end{itemize}

Correlations between the Asymmetric and Symmetric series are rather
constant throught the troposphere, fluctiating between 0.39 and 0.45
(Figure \ref{fig:cor-lev}). Futhermore, the cross-correlation of each
series across levels --shown in Figure \ref{fig:cross-correlation}-- are
high in the trosposphere (greater than 0.9) for both indices. This
suggests that both the Asymmetric and the Symemtric component of the
tropospheric SAM are highly vertically coherent, both in their
individual evolution and their temporal relationship. This is to be
expected since the SAM is mostly equivalent barotropic (citaaaa).

In the stratosphere the situation is different. As can be seen in Figure
\ref{fig:cor-lev}, the relationship between the Asymmetric and Symmetric
indices varies with height above 100 hPa. It starts to decreese right
over the tropopause, reaches a minium of 0.21 at 20 hPa and then it
increases again monotonically with height up to the uppermost level of
the reanalysis. The cross-correlation across levels in the stratosphere
is generally weaker than in the troposphere (Figure
\ref{fig:cross-correlation}). Furthermore, above 100 hPa, the
cross-correlation decreases more rapdily with height for the Symmetric
SAM than for the Asymmetric SAM as evidenced by the wider dark red areas
near the diagonal in Figure \ref{fig:cross-correlation}b) vs.~Figure
\ref{fig:cross-correlation}c). Moreover, the stratospheric Symmetric SAM
seems to be slightly more connected to the trosposphere than the
Asymmetric SAM; this can be seen by the lower correlation values in the
top right quadrant of Figure \ref{fig:cross-correlation}b) in comparison
with Figure \ref{fig:cross-correlation}c).

\begin{figure}
\includegraphics{cor-lev-1} \caption[Correlation between the Symmetric and Asymmetric SAM at each level]{Correlation between the Symmetric and Asymmetric SAM at each level.}\label{fig:cor-lev}
\end{figure}

\begin{figure*}
\includegraphics{cross-correlation-1} \caption[Cross correlation between levels of the Full, Asymmetric and Symmetric SAM]{Cross correlation between levels of the Full, Asymmetric and Symmetric SAM.}\label{fig:cross-correlation}
\end{figure*}

Figure \ref{fig:cross-correlation}a) show the cross-correlation across
levels for the Full SAM index. \#FIXME

\begin{figure}
\includegraphics{trends-1} \caption[Trends for each index at each level]{Trends for each index at each level. Shading indicates the 95\% confidence interval.}\label{fig:trends}
\end{figure}

Figure \ref{fig:trends} shows normalised decadal trends for each index
for the whole period along with a 95\% interval in shading. There is a
statistically significant increase in positive SAM in the tropostphere
(panel a), which has been already documented in other studies (e.g.
\citet{fogt2020}). Panels b and c of Figure \ref{fig:trends} show that
this increase is evident only in the Symmetric component. This
distinction should prove useful when attributing trends in other
variables such as temperature and precipitation to trends in the SAM.

\includegraphics{unnamed-chunk-1-1} \includegraphics{unnamed-chunk-1-2}
\includegraphics{unnamed-chunk-1-3} \includegraphics{unnamed-chunk-1-4}

\subsection{Spatial patterns}

To undertand the spatial patterns associated with both indeces, we
regressed monthly geopotential anomalies into both indeces using
multiple regression (Figure A6 illustrates the difference between
computing two simple regressions and one multiple regression).

\begin{figure*}
\includegraphics{2d-regr-1} \caption[Regression patterns of geopotential height at 30, 300 and 700 hPa with the Full, Asymmetric and Symmetric SAM]{Regression patterns of geopotential height at 30, 300 and 700 hPa with the Full, Asymmetric and Symmetric SAM. The regression patterns for Asymmetric and Symmetric SAM are the result of one multiple regression using both indices, not of two simple regressions involving each index by itsef.}\label{fig:2d-regr}
\end{figure*}

Figure \ref{fig:2d-regr} shows the spatial year-long regression for
selected levels. In the troposphere the Full annular mode is clearly
``contaminated'' with well known zonal asymemtries (panels g and j)
which are succesfully sepparated by our methodology (panels h, i, k and
l). In the stratosphere, the spatial pattern associated with the Full
SAM is much more clearly dominated by a zonally symmetric, monopolar
structure (panels a and d) that is, however, not perfectly centered in
the south pole. The monopoloe obtained by multiple regression with the
Asymmetric and Symmetric SAM (panels c and f in Figure
\ref{fig:2d-regr}) is much more symmetric and the shift from total
symmetry is captured by the regression pattern of the Asymmetric SAM as
a wave-1 pattern (panels b and e).

\begin{figure}
\includegraphics{wave-amplitude-1} \caption[Planteray wave amplitude for the regression patterns at 50 and 700 hPa]{Planteray wave amplitude for the regression patterns at 50 and 700 hPa.}\label{fig:wave-amplitude}
\end{figure}

The amplitude of each zonal wave number at each latitude at 50 hPa and
700 hPa is shown in Figure \ref{fig:wave-amplitude}, where wave number
zero represents the zonal mean. Comparing between rows, this Figure
quantifies the relatively clean separation between the zonally symmetric
and zonally asymmetric structures, as its evident how the mixture of
waves of the Full field (first row) is very similar to the sum of the
waves of the Asymmetric and Symmetric field (second and third row,
respectively). The second row of Figure \ref{fig:wave-amplitude} shows
that the Asymmetric SAM is overwhelmingly dominated by wave 1 in the
stratosphere (panel b), while in the stratosphere it is composed of
zonal waves 3 to 1 in decreasing level of importance.

\begin{figure}
\includegraphics{vertical-regression-1} \caption[Asymmetric coefficient of the multiple regression of mean monthly geopotential height anomalies between 65 and 40 South]{Asymmetric coefficient of the multiple regression of mean monthly geopotential height anomalies between 65 and 40 South. (this caption needs some love)}\label{fig:vertical-regression}
\end{figure}

From Figure \ref{fig:2d-regr} it appears that the vertical structure of
the Asymmetric SAM is equivalent barotropic in the troposphere but
baroclinic in the stratosphere. Anomalies are centerd in the same
locations in the troposphere (panels h and k), but show westerly
displacement in the stratosphere (panesl b and e). This is expanded in
Figure \ref{fig:vertical-regression}, which shows a vertical crossection
of the regression coefficient corresponding to the middle row of Figure
\ref{fig:2d-regr}, area-weighted averaged between 65 and 40 degrees
South. Below 100 hPa, anomalies are completely vertical, while above
they show an important westerly tilt with height.

\subsection{Impacts}

\subsubsection{Temperature}

\begin{sidewaysfigure}
\includegraphics{regr-air-season-1} \caption[Regression pattern of surface temperature with Asymmetric and Symmetric SAM]{Regression pattern of surface temperature with Asymmetric and Symmetric SAM. P-values smaller than 0.05 (controlling for Flase Detection Rate) as hatched areas. Gray areas have more than 15\% of missing data.}\label{fig:regr-air-season}
\end{sidewaysfigure}

Figure \ref{fig:regr-air-season} shows regression coefficients of each
index at 700 hPa with surface temperature for each trimester. It is
evident that the Asymmetric and Symmetric SAM indices are associated
with overall distinct temperature patterns which can be obscured when
using the Full SAM index. The Symmetric SAM signal is weaker than the
Asymmetric SAM, as evidenced by the relatively smaller and les
sstatistically significant regression coefficients in row 3 of Figure
\ref{fig:regr-air-season} compared with row 2.

In DJF (column a), the strong negative signal in the tropical Pacific in
panel a.1 is mostly associated with the Asymmetric component (panel
a.2), as is it largely absent in the Symmetric component (panel a.3).
Furthermore, the Asymmetric SAM is also asociated with low temperature
anomalies in the Indian ocean, but this signal is obscured by the
Symmetric variability and thus lost in the Full SAM. Over the
continents, the Asymmetric SAM is assoiated with negative temperature
anomalies which, again, mostly disappear in the Full SAM regression.

The patterns seen in MAM and JJA (columns b and c) are not robustly
significant in the sense that there are no areas with p-values below
0.05 when controlling for FDR following \citet{wilks2016}. Nevertheless,
it is interesting to note that in both trimesters, the sign of the
regression is consistently flipped between the Asymmetric and Symmetric
regressions. In South America, for example, the Asymmetric SAM is
associated with positive temperature anomalies in MAM and negative
temperature anomales in JJA, while the oposite is the case for the
Symemtric SAM.

Finally, in SON (column d), there is no significant temperature signal
associated with the Symmetric SAM (panel d.3), while the Asymmetric SAM
shows a relatively robust signal in the equatorial Pacitic, Australia,
and even Southeast South America. This strong signals are reduced in
intensity in panel a.3.

\subsubsection{Precipitation}

\begin{sidewaysfigure}
\includegraphics{pp-regr-global-1} \caption[Regression pattern of precipiation with Asymmetric and Symmetric SAM]{Regression pattern of precipiation with Asymmetric and Symmetric SAM. P-values smaller than 0.05 (controlling for Flase Detection Rate) as hatched areas.}\label{fig:pp-regr-global}
\end{sidewaysfigure}

\acknowledgments

CMAP Precipitation data provided by the NOAA/OAR/ESRL PSL, Boulder,
Colorado, USA, from their Web site at https://psl.noaa.gov/

NOAA Global Surface Temperature (NOAAGlobalTemp) data provided by the
NOAA/OAR/ESRL PSL, Boulder, Colorado, USA, from their Web site at
https://psl.noaa.gov/

\bibliography{AsymSAM}

\newpage

\appendix

\appendixtitle{Extra figures}

\begin{figure}
\includegraphics{A1-1} \appendcaption{A1}{Lag-correlation between Symmetric and Asymmetric SAM at each level.}\label{fig:A1}
\end{figure}

\begin{figure}
\includegraphics{ccf-levels-1} \caption[Cross-correlation functions for each index and two differnet base levels]{Cross-correlation functions for each index and two differnet base levels.}\label{fig:ccf-levels}
\end{figure}

\begin{figure}
\includegraphics{A2-1} \appendcaption{A2}{Fourier spectrum of each timeseries. The shading indicates de 95\% area derived by fitting an AR process to each series and bootstrapping 5000 simulated samples.}\label{fig:A2}
\end{figure}

\begin{figure}
\includegraphics{A3-1} \appendcaption{A3}{Autocorrelation functions of each timeseries}\label{fig:A3}
\end{figure}

\begin{figure}
\includegraphics{A5-1} \appendcaption{A5}{Regression pattern of precipiation with Asymmetric and Symmetric SAM. P-values smaller than 0.05 (controlling for Flase Detection Rate) as hatched areas.}\label{fig:A5}
\end{figure}

\begin{figure}
\includegraphics{A6-1} \appendcaption{A6}{Regressions maps resulting from performing one multiple regression (a. and b.) and from performing two simple regressions (c. and d.)}\label{fig:A6}
\end{figure}

\begin{figure}
\includegraphics{A7-1} \appendcaption{A7}{ }\label{fig:A7}
\end{figure}

\begin{figure}
\includegraphics{A8-1} \appendcaption{A8}{Decadal trends of SAM indices for each season. }\label{fig:A8}
\end{figure}


\end{document}
